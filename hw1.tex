\section{Linear Density Model}
\textbf{A useful (albeit not terribly realistic) model for a homogeneous composition star may be obtained by assuming that the density is a linear function of the radius (see Stein, 1966). 
Thus assume that $\rho(r)=\rho_c(1 - r/R)$, where $\rho_c$ is the central density and $R$ is the total radius where zero boundary conditions $P(R) = T(R) = 0$ apply.}

\subsection{Central density}
The mass contained within a spherical shell is given by $dM_r= 4\pi r^2\rho dr$. 
Since the density is linear with $r$, then 
\begin{equation}
    dM_r= 4\pi r^2\rho_c(1-r/R)dr.
\end{equation} 

We can integrate over all the radius to find that the total mass, $M$ is
\begin{align*}
    M&=\int^R_0 4\pi r^2\rho_c(1-\frac{r}{R})dr\\
    &=4\pi\rho_c\int^R_0\left(r^2-\frac{r^3}{R}\right)dr\\
    &=4\pi\rho_c \left[\frac{r^3}{3} -\frac{r^4}{4R}\right]^R_0\\
    &=4\pi\rho_c \left(R^3 -\frac{R^4}{4R}\right)=4\pi\rho_cR^3/12\\
    &=\frac{\pi\rho_cR^3}{3}
\end{align*}

Then we can find that the central density is
\begin{equation}
    \rho_c = \frac{3M}{\pi R^3}
    \label{eq:centralDensity}
\end{equation}

%==============================================================

\subsection{Pressure and Central Pressure}
% Use the equation of hydrostatic equilibrium and zero boundary conditions to find pressure as a function of radius. Your answer will be of the form $P(r) = P_c \times (\text{polynomial in }r/R)$. 
% What is $P_c$ in terms of $M$ and $R$? Express $P_c$ numerically with $M$ and $R$ in solar units (i.e., $M_\odot$ and $R_\odot$).
The hydrostatic equilibrium is given when the pressure and the gravitational force are in balance. 
Assuming a spherical symmetry the hydrostatic equilibrium equation can be derived from the equation of motion and assuming there is no acceleration so that $dP/dr = -GM_r\rho/r^2$.

Similarly as how we find the total mass, we can find that $M_r=4\pi\rho_c(r^3/3 - r^4/4R)$.
Substituting the central density with our result in equation \ref{eq:centralDensity},
\begin{align*}
    M_r&= 4\pi\rho_c\left(\frac{r^3}{3} - \frac{r^4}{4R}\right)\\
    &= 4\pi \left(\frac{3M}{\pi R^3}\right)\left(\frac{r^3}{3} - \frac{r^4}{4R}\right)\\
    &= \frac{4Mr^3}{R^3}\left(1-\frac{3r}{4R}\right), \numberthis \label{eq:massLinearModel}
\end{align*}

we can find an expression for the pressure
\begin{align*}
    \frac{dP}{dr} &= -\frac{GM_r\rho}{r^2} \\
     &=-\frac{G}{r^2}\frac{4Mr^3}{R^3}\left(1-\frac{3r}{4R}\right)\frac{3M}{\pi R^3}\left(1-\frac{r}{R}\right)\\
     &= -\frac{12GM^2r}{\pi R^6}\left(1-\frac{7r}{4R}+\frac{3r^2}{4R^2}\right)\\
     &=-\frac{12GM^2}{\pi R^5}\left(\frac{r}{R}-\frac{7r^2}{4R^2}+\frac{3r^3}{4R^3}\right)\\
\end{align*}

Integrating we have,
\begin{align*}
    \int^{P(r)}_{P_c}dP &= \int^{r}_0-\frac{12GM^2}{\pi R^5}\left(\frac{r}{R}-\frac{7r^2}{4R^2}+\frac{3r^3}{4R^3}\right)dr\\
    P(r) - P_c &= -\frac{12GM^2}{\pi R^5}\left(\frac{r^2}{2R}-\frac{7r^3}{12R^2}+\frac{3r^4}{16R^3}\right) \numberthis \label{eq:pressureExpanded}
\end{align*}

Using the boundary conditions $P(R)=0$, we can solve for $P_c$,

\begin{align*}
    P_c &= \frac{12GM^2}{\pi R^5}\left(\frac{R^2}{2R}-\frac{7R^3}{12R^2}+\frac{3R^4}{16R^3}\right)\\
    &= \frac{12GM^2}{\pi R^5}\left(\frac{5R}{48}\right)=\frac{5}{4}\frac{GM^2}{\pi R^4} \numberthis \label{eq:pressureCentral}\\
\end{align*}

The central pressure in solar units\footnote{ $G=\SI{6.6726e-8}{\cubic\cm\per\g\per\square\s}$, $M_\odot = \SI{1.9891e33}{\g}$, and $R_\odot = \SI{6.96e10}{\cm}$} is given by,
\begin{align*}
    P_c &= \frac{5}{4}\frac{GM_\odot^2}{\pi R_\odot^4}\left(\frac{M}{M_\odot}\right)^2\left(\frac{R}{R_\odot}\right)^{-4}\\
    &= 4.48\times 10^{15}\left(\frac{M}{M_\odot}\right)^2\left(\frac{R}{R_\odot}\right)^{-4}\si{\g\per\cm\per\square\s}
\end{align*}

Then substituting on equation \ref{eq:pressureExpanded}, we can get the final expression for the pressure as,
\begin{align*}
    & P(r) = -\frac{12GM^2}{\pi R^5}\left(\frac{r^2}{2R}-\frac{7r^3}{12R^2}+\frac{3r^4}{16R^3}\right) + P_c\\
    &= -\frac{12GM^2}{\pi R^5}\left(\frac{r^2}{2R}-\frac{7r^3}{12R^2}+\frac{3r^4}{16R^3}\right) + \frac{5}{4}\frac{GM^2}{\pi R^4}\\
    &= \frac{5}{4}\frac{GM^2}{\pi R^4}\left[-\frac{48}{5 R}\left(\frac{r^2}{2R}-\frac{7r^3}{12R^2}+\frac{3r^4}{16R^3}\right) +1 \right]\\
    &= P_c\left[1-\frac{24}{5}\left(\frac{r^2}{R^2}-\frac{7r^3}{6R^3}+\frac{3r^4}{8R^4}\right)\right]\numberthis\label{eq:pressureLinearModel}
\end{align*}

 
%==============================================================

\subsection{}
% In this model, what is the central temperature, $T_c$? (Assume an ideal gas.) 
% Compare this result to that obtained for the constant-density model.
Assuming an ideal gas, we can find that the temperature is given by 
\begin{equation}
    T=\frac{P}{n\kappa}
\end{equation}
where $n\equiv \rho/\mu m_A$ and $\kappa$ is the number density and where $m_A$ is the atomic mass unit ( \SI{1.6605402e-24}{\g}).

Then the central temperature $T_c$ will be given by
\begin{align*}
    T_c &=\frac{P_c\mu m_A}{\rho_c\kappa}\\
    &=\frac{\frac{5}{4}\frac{GM^2}{\pi R^4} \mu m_A}{\frac{3M}{\pi R^3}\kappa}\\
    &= \frac{5}{12}\frac{GM}{R}\frac{\mu m_A}{\kappa}
\end{align*}
and numerically in solar units as
\begin{align*}
    T_c &= \frac{5}{12}\frac{GM_\odot}{R_\odot}\frac{\mu m_A}{\kappa}\left(\frac{M}{M_\odot}\right)\left(\frac{R}{R_\odot}\right)^{-1}\\
    &= 9.56\times 10^6 \mu \left(\frac{M}{M_\odot}\right)\left(\frac{R}{R_\odot}\right)^{-1}\si{\per\kelvin}
\end{align*}

The central pressure is from equation \ref{eq:pressureCentral} and the central density is from equation \ref{eq:centralDensity}.

\textbf{Why is the central pressure higher for the linear model whereas the central temperature is lower?}

\begin{table*}[]
    \centering
    \begin{tabular}{c|ccc}
    \toprule
         & $P_c$ & $T_c$ & $\rho_c$ \\
         & \si{\g\per\cm\per\square\s} & \si{\per\kelvin} & \si{\g\per\cubic\cm}\\
    \midrule
         Constant Model 
            &  $1.34\times 10^{15}\left(\frac{M}{M_\odot}\right)^2\left(\frac{R}{R_\odot}\right)^{-4}$ 
            &  $1.15\times 10^{7}\mu\left(\frac{M}{M_\odot}\right)\left(\frac{R}{R_\odot}\right)^{-1}$
            &  $1.408\left(\frac{M}{M_\odot}\right)\left(\frac{R}{R_\odot}\right)^{-3}$\\
         Linear Model 
            & $4.48\times 10^{15}\left(\frac{M}{M_\odot}\right)^2\left(\frac{R}{R_\odot}\right)^{-4}$
            & $9.56\times 10^{6}\mu\left(\frac{M}{M_\odot}\right)\left(\frac{R}{R_\odot}\right)^{-1}$
            & $5.634\left(\frac{M}{M_\odot}\right)\left(\frac{R}{R_\odot}\right)^{-3}$\\
    \bottomrule
    \end{tabular}
    \caption{Comparison of the central pressure, $P_c$ and the central temperature, $T_c$ (in solar units) between the constant density model and the linear density model.}
    \label{tab:comparisonLinearConstantModel}
\end{table*}

From the ideal gas relation we have that $ P \propto T \rho$ so pressure is proportional to the density and the temperature is inversely proportional to the density. For the linear model the density is $\rho_c = 3M/\pi R$ which is higher than the constant model, which has $\rho_c = 3M/4\pi R$. This causes that the linear model has a higher pressure and a lower temperature than the constant model. This is in accordance to our results shown in Table \ref{tab:comparisonLinearConstantModel}.

%==============================================================

\subsection{}
\textbf{Verify that the virial theorem is satisfied for the entire star and write down an explicit expression for the potential energy of the star, $\Omega$ (i.e., what is $q$ of Eq. 1.8?).}

The virial theorem equation is given by 
\begin{equation}
    \frac{1}{2}\frac{d^2I}{dt^2} = 2K + \Omega
\end{equation}
where K is the kinetic energy and $\Omega$ is the potential energy. We say that the system satisfied the virial theorem when $\frac{1}{2}\frac{d^2I}{dt^2} = 0 $ so that $ 2K = -\Omega$.

We can find the kinetic energy by  
\begin{align*}
    &2K = \int_M\frac{3P(r)}{\rho(r)}dM_r = \int_R 3P(r)4\pi r^2 dr\\
    &= 12\pi P_c\int_R\left[1-\frac{24}{5}\left(\frac{r^2}{R^2}-\frac{7r^3}{6R^3}+\frac{3r^4}{8R^4}\right)\right]r^2\\
    &= \frac{60}{4}\frac{GM^2}{R^4} \left[\frac{r^3}{3}-\frac{24r^5}{25R^2}+\frac{28r^6}{30R^3}-\frac{9r^7}{35R^4}\right]_0^R\\
    &= \frac{60}{4}\frac{GM^2}{R}\left[\frac{1}{3}-\frac{24}{25}+\frac{28}{30}-\frac{9}{35}\right]\\
    &= \frac{26}{35}\frac{GM^2}{R}\numberthis\label{eq:kineticEnergy}
\end{align*}

The potential energy is given by
\begin{align*}
    &\Omega = -\int_M \frac{GM_r}{r}dM_r = -\int_R \frac{GM_r}{r}4\pi r^2 \rho dr\\
    &= -\int_R 4\pi G r \left(\frac{4Mr^3}{R^3}\left(1-\frac{3r}{4R}\right)\right)\left(\frac{3M}{\pi R^3}(1 - r/R)\right)\\
    &= -\frac{48GM^2}{R^6}\left[\frac{r^5}{5}-\frac{7r^6}{24R}+\frac{3r^7}{28R^2}\right]^R_0\\
    &= -\frac{48GM^2}{R}\left[\frac{1}{5}-\frac{7}{24}+\frac{3}{28}\right]\\
    &= -\frac{26}{35}\frac{GM^2}{R}
\end{align*}

Thus we can verify that
\begin{align*}
    2K = - U\\
    \frac{26}{35}\frac{GM^2}{R} = -\left[-\frac{26}{35}\frac{GM^2}{R}\right]
\end{align*}
is true.

%==============================================================
\clearpage
%==============================================================
\section{}
\textbf{Consider a star with density structure $\rho(r)=\rho_c(1 - r/R)$.
Assume that the star is pure ionized Hydrogen, that the ions behave as a perfect gas, that the electrons are non-relativistic and degenerate with pressure described by $P_d = C\rho^{5/3}$, where $C$ is a constant (see Eqs. 3.51 and 3.53; dropping all terms but the first in the polynomial) and that the central pressure (as calculated above) is due to a combination of gas and degeneracy pressure.
Please find an expression for the maximum temperature, $T_{c;m}$, at the center.
This expression should be independent of the central density and should only depend on $M$ and fundamental constants. A proof of the obtained $T_{c;m}$ is indeed a maximum should be given.}

The pressure is due to a combination of the gas, from ions, and degeneracy pressure so that
\begin{equation*}
    P= P_i + P_d = n_i\kappa T + C\rho^{5/3}
\end{equation*}
Solving for the temperature T, we have
\begin{equation}
    T= \frac{P - C\rho^{5/3}}{n_i\kappa}=\frac{P - C\rho^{5/3}}{\rho\kappa}\mu_i m_a
\end{equation}
given that $n_i = \rho / \mu_i m_a$. 
The central temperature can then be found by using the central pressure \ref{eq:pressureCentral} and central density from problem 1, so that
\begin{align}
    T_c &= \frac{P_c - C\rho_c^{5/3}}{\rho_c\kappa}\mu_i m_a\\
    &= \frac{\mu_i m_a}{\rho_c\kappa}\left(\frac{5}{4}\frac{GM^2}{\pi R^4} - C\rho_c\right)
\end{align}


%==============================================================
\clearpage
%==============================================================
\section{}
\textbf{Mass loss from winds in hot luminous stars. The mass loss rate $\Dot{M}$ in a stellar wind from a hot, massive star of mass $M$, radius $R$, and luminosity L obeys the semi-empirical relation
\begin{equation}
    \log(\Dot{M}v_\infty R^{1/2}) = -1.37 + 2.07\log(L/10^6),
\end{equation}
where $M$, $R$, and $L$ are measured in solar units, $\Dot{M}$ is measured in $M_\odot yr^{-1}$, and $v_\infty$ is the terminal velocity of the wind (far from the star) in \si{\kilo\meter\per\s}. 
This terminal velocity $v_\infty$ is found to be roughly proportional to the escape velocity $v_{esc}$ at the stellar surface,
\begin{equation}
    v_{esc}= \sqrt{2(1-\Gamma_e)GM/R}
\end{equation}
The factor of $(1-\Gamma_e)$ rises from the levitating effect of radiation pressure at the photosphere, which effectively lowers the escape velocity.
For stars with $T \gtrsim$ \SI{2.1e4}{\kelvin}, it turns out that $v_\infty = v_{esc} /2.6$.}
%==============================================================

\subsection{}
\textbf{Neglecting radiation pressure (i.e., setting $\Gamma_e = 0$), please plot $\log\Dot{M}$ (in units of $M_\odot yr^{-1}$ versus $M$ (in the $20-100 M_\odot$ range) for luminosities of $10^5 L_\odot$, $3\times 10^5 L_\odot$, $10^6 L_\odot$, and $2\times 10^6 L_\odot$.
In the same plot, please also include the $\log\Dot{M} - M$ relation for main sequence stars (The $L/L_\odot \approx (M/M_\odot)^3$ scaling relation as learned earlier may be used. e.g., section 1.6 of HKT).}

To create this plot we had to use the gravitational constant in units of solar units
\begin{equation}
    G = 6.6740\times 10^{11}\quad \si{\km^2\s^{-2}}M_\odot^{-1}R_\odot
\end{equation}
so that $v_{esc}$ and $v_\infty$ will be in units of \si{\km\per\s}.

Solving for the mass loss rate, we have
\begin{equation}
    \log\Dot{M} = -1.37 + 2.07\log(L/10^6)-\log(v_\infty R^{1/2})
    \label{eq:massLossRate}
\end{equation}

Figure \ref{fig:problem3Fig} shows the $\log\Dot{M}$ versus $M$ for luminosities of $10^5 L_\odot$, $3\times 10^5 L_\odot$, $10^6 L_\odot$, and $2\times 10^6 L_\odot$ given by the solid lines, assuming $R =1 $ in solar units and $\Gamma_e = 0$, since we are neglecting radiation pressure. We can note that the mass loss rate decreases with M and increases with luminosity. Additionally, we can se that the blue solid line is the mass loss rate for when $L\propto M^3$ which is the case for zero age main sequence stars. We can note that for a ZAMS the mass loss rate increases with mass which is the opposite of what we get when including mass loss from winds.  

\begin{figure}
    \centering
    \includegraphics{hw1problem3fig1.pdf}
    \caption{Plot shows the mass loss rate due to winds in hot luminous stars using the equation \ref{eq:massLossRate}. The solid lines are when we are neglecting radiation pressure and the dash lines are with radiation pressure. }
    \label{fig:problem3Fig}
\end{figure}


%==============================================================
\subsection{}
\textbf{Now consider the effects of radiation pressure, by using
\begin{equation}
    \Gamma_e = \frac{\kappa L}{4\pi c GM},
\end{equation}
with all quantities in cgs units, and using the electron scattering opacity for the winds of hot stars with $\kappa = 0.3$ \si{\centi\meter\per\g}.
What is the effect on the mass loss rates?}

Figure \ref{fig:problem3Fig} shows the mass loss rate considering the radiation pressure in dash lines. We can see that the mass loss rate still decreases with M but it seem to be exponential which at larger M it tends to the mass loss rate without radiation pressure.



%==============================================================
\clearpage
%==============================================================
\section{}
\textbf{Photon mean free paths are very short except in the outermost
layers of a star.
This means that photons must take a very long time to escape from a star.
To estimate this time, assume that a photon is created at the center of the star and undergoes a series of random walk scatterings until it reaches the surface.
The mean free path associated with each scattering is $\lambda_{phot} = (n_e\sigma_e)^{-1}$, where $\sigma_e$ is the Thomson scattering cross section $\sigma_e \approx 0.7 \times 10^{-24}$ \si{\square\centi\meter}. 
Assume constant density and $\lambda_{phot}$.}
%==============================================================
\subsection{}
\textbf{Using 1-D random walk arguments, show that $L\approx R^2 / \lambda_{ph}$ is the mean total distance a photon travels if it starts its scattering career at the stellar center and eventually ends up at the surface at $R$.}

Using random walk arguments, we have that the average distance after N steps is equal to
\begin{equation*}
    \frac{R}{\lambda_{phot}}\approx\sqrt{N}
\end{equation*}
and the total distance travelled is 
\begin{equation*}
    L \approx N \lambda_{phot} = \frac{R^2}{\lambda_{phot}^2}\lambda_{phot}=\frac{R^2}{\lambda_{phot}}
\end{equation*}

%==============================================================
\subsection{}
\textbf{Since the photons travel at the speed of light, $c$, find the time, $t$, required for the photon to travel from the stellar center to the surface at $R$.}

We know that the velocity is equal to the distance over time so we can find that the time is
\begin{equation*}
    t = \frac{L}{c} = \frac{R^2}{\lambda_{phot}c} = \frac{R^2n_e\sigma_e}{c}
\end{equation*}
 where $v=c$ for the photons.

%==============================================================
\subsection{}
\textbf{Give a quantitative estimate for $t$, in units of years, for a star of mass $M/M_odot$ and radius $R/R_odot$. Assume solar composition.}

We know that 
\begin{equation}
    n_e = \frac{\rho N_A}{\mu_e} = \frac{\rho N_A (1+X)}{2}=\frac{3MN_A(1+X)}{8\pi R^3}
\end{equation}
where $\mu_e = 2/(1+X)$ where X is the mass fraction of the ions and the density, $\rho$, is given by $\rho = 3 M / 4\pi R^3$ from the constant density model.

Then we have that the time is given by
\begin{equation}
    t = \frac{R^2\sigma_e}{c}\frac{\rho N_A (1+X)}{2}=\frac{3MN_A(1+X)}{8\pi R^3}
\end{equation}

Evaluating all constants, converting the mass and radius in solar units and using $X=0.7$ since we are assuming solar composition, then we get that

\begin{equation}
    t = \SI{8.154e10}{\s} = 2585 \frac{M}{M_\odot}\frac{R}{R_\odot}^{-1} \text{yrs}
\end{equation}

%==============================================================
\clearpage
%==============================================================
\section{}
\textbf{Let's look at corrections to Maxwell- \\ Boltzmann statistics due to weak electron degeneracy (assuming a non-relativistic \\case).
Suppose that $\mu/\kappa T$ is still much less than $-1$, but we wish to include some effects of Fermi-Dirac statistics.
In other words, what are the effects due to the $+1$ in the distribution function?}
%==============================================================
\subsection{}
\textbf{If the exponential term in Eq. 3.9 is still large, then you can use the expansion $1/(a + 1) \approx (1 - 1/a)/a$ to first order in $a$. Assume that $\mu/\kappa T$ is still given by
\begin{equation}
    e^{\mu/\kappa T} = \frac{n_0 h^3}{2(2\pi m_e \kappa T)^{3/2}} \equiv K
    \label{eq:bigK}
\end{equation}
where $n_0$ is the electron number density in the pure Maxwell-Boltzmann limit.
Show that the number density $n$ for weak degeneracy is
\begin{equation}
    n=n_0\left[1-2^{-3/2}K\right]
\end{equation}}

Equation 3.9 from HKT is
\begin{equation*}
    n(p)=\frac{1}{h^2}\Sum_j\frac{g_j}{\exp{-\mu - \epsilon_j + \epsilon(p)}}
\end{equation*}

For this case we can use $g=2$ for fermions, $\epsilon=0$ since we are assuming there's only one energy level, and $\epsilon(p)=p^2/2m$ considering the non-relativistic case. Then
\begin{equation*}
    n(p) = \frac{2}{h^3}\frac{1}{\exp{-\mu/\kappa T}\exp{p^2/2m\kappa T}}
\end{equation*}

Using the expansion $1/(a + 1) \approx (1 - 1/a)/a$ to first order in $a$ and the relation \ref{eq:bigK}, we have that
\begin{equation}
    n(p) = \frac{2K}{h^3}\left(e^{p^2/2m\kappa T}-Ke^{p^2/m\kappa T}\right)
\end{equation}

Then, the number density is given by
\begin{align*}
    n = 4\pi \int_p p^2 n(p) dp\\
    n =  \frac{8\pi K}{h^3}\int_p p^2\left(e^{p^2/2m\kappa T}-Ke^{p^2/m\kappa T}\right)dp\\
\end{align*}
We can note that we have two terms that are similar to $x^n e^{x^{-2}}$ which immediately calls for the use of gamma functions.
For the first integral, we use the following transformation
\begin{align*}
    x = p/\sqrt{2m\kappa T}\\
    dx = dp / \sqrt{2m\kappa T}\\
\end{align*}
and for the second integral, we use
\begin{align*}
    x = p/\sqrt{m\kappa T}\\
    dx = dp / \sqrt{m\kappa T}\\
\end{align*}

Substituting on $n$, we get
\begin{align*}
    n &= \frac{8\pi K}{h^3}\left[(2m\kappa T)^{3/2}\int_0^\infty x^2 e^{x^{-2}} dx \\ 
    &\qquad - K(m\kappa T)^{3/2}\int_0^\infty x^2 e^{x^{-2}} dx \right]\\
    n &= \frac{8\pi K(m\kappa T)^{3/2}}{h^3}\left[2^{3/2}\frac{\sqrt{\pi}}{4}- K\frac{\sqrt{\pi}}{4}\right]\\
    n &= \frac{8\pi(m\kappa T)^{3/2}}{h^3}\left(\frac{n_0 h^3}{2(2\pi m_e \kappa T)^{3/2}}\right)\left[2^{3/2}-K\right]\\
    n &= n_0\left[1 - 2^{-3/2}K\right]
\end{align*}

%==============================================================
\subsection{}
\textbf{Show that the new pressure is
\begin{equation}
    P = n_0\kappa T \left[1-2^{-5/2}K\right]
\end{equation}}

Using the same value of $n(p)$, $v=p/m$ and the same substitution of gamma function we can find the the pressure is
\begin{align*}
    P &= \frac{4\pi}{3m}\int_p p^4 n(p) dp\\
    &= \frac{4\pi}{3m}\int_p p^4 \frac{2K}{h^3}\left(e^{p^2/2m\kappa T}-Ke^{p^2/m\kappa T}\right) dp\\
    &= \frac{8\pi K(m\kappa T)^{2/3}}{3h^3m}\left[2^{5/3}\int_0^\infty x^4 e^{x^{-2}} dx - K\int_0^\infty x^4 e^{x^{-2}} dx\right]\\
    &= \frac{8\pi(m\kappa T)^{2/3}}{3h^3m}\left(\frac{n_0 h^3}{2(2\pi m_e \kappa T)^{3/2}}\right)\left[2^{5/2}-K\right]\\
    &= n_0\kappa T \left[1-2^{-5/2}K\right]
\end{align*}


%==============================================================
\subsection{}
\textbf{In this case of weak degeneracy, what is $P(n)/P_{id}(n)$ to the lowest order in $K$, where $P_{id}(n) = n\kappa T$ is the pressure that an ideal gas of density $n$ would have?
Have a brief discussion of the meaning of the result.}

Substituting the $n$ that we found previously into the equation of an ideal gas, we get 
\begin{equation*}
    P_{id}(n) = n_0\kappa T \left[1-2^{-3/2}K\right]
\end{equation*}
 Then the ratio is
\begin{align*}
    \frac{P(n)}{P_{id}(n)} &= \frac{n_0\kappa T \left[1-2^{-5/2}K\right]}{n_0\kappa T \left[1-2^{-3/2}K\right]}\\
    &= \frac{\left[1-2^{-5/2}K\right]}{\left[1-2^{-3/2}K\right]}.
\end{align*}

This tell us that the pressure considering a weak degeneracy is bigger than that from an ideal gas. We can also note that there will be a discontinuity if $K=2^{3/2}$ and it will be zero if $K=2^{5/2}$. 

