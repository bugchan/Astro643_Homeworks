\section{}(10 points)
Estimate the gravitational bounding energy of a white dwarf and compare this energy to that of a neutron star. Would the energy released from the nuclear burning of a 1.4 $M_\odot$ Carbon white dwarf unbound it? Could the energy unbound a neutron star
of a similar mass?


%==============================================================
\section{}(20 points)
Consider the hypothetical evolution of a star of initial mass $M_0$.
The star's core grows in mass as a result of nuclear burning.
The nuclear processes release an amount of energy $Q$ per gram of burnt material. 
The star loses mass (by means of a stellar wind) at a rate proportional to its constant luminosity $L$, $\Dot{M} = \alpha L$.
\subsection{}
Find the mass of the core as a function of time, $M_c(t)$, assuming that $M_c(0) = 0$.
\subsection{}
Assuming that the initial mass of the envelope is $M_e(0) = M_0$, what is the core mass when the envelope mass vanishes?
\subsection{}
Calculate the upper limit of $M_0$, for which the star will become a white dwarf (considering $M_{Ch} = 1.4M_\odot$), given $Q = \SI{5e18}{\erg\per\g}$ (from turning solar composition
into carbon and oxygen) and $\alpha = \SI{1e-18}{g\per\erg}$.


%==============================================================
\section{}(40 points)
In the following assume the visual magnitude of the Sun as seen from the Earth is $m_{V,\odot} = -26.5$, and the observed color index of the Sun is $(B-V)_\odot = 0.60$.
\subsection{}
You observe a star with a spectrum that appears to be identical to that of the Sun.
The star has an observed visual magnitude $m_V = 14$. How far away is the star (in parsecs) assuming there is no dust along the line of sight?
\subsection{}
Someone later observes that the color of the star is $B-V = 1.10$. 
Recalculate the distance assuming a standard extinction law with $R = 3.1$.
\subsection{}
What would you expect the J magnitude of this star to be?
You will need to look up the intrinsic colors of a G2 V star somewhere, for example, Kenyon \& Hartmann (1995, ApJS 101, 117) and take into account any corrections for reddening at J.
If the J-magnitude turned out to be brighter than expected, what explanation might you put forward?
\subsection{}
Oops. Now a new observation shows that in fact the star is a double-lined spectroscopic binary, with both components having identical spectral types.
The half-amplitude of the radial velocity curves is \SI{20}{\kilo\meter\per\second} for each component, and the period is 25 days.
Find the inclination of the orbit (0 degrees means the orbit is in the plane of the sky).
Is the system an eclipsing binary?
\subsection{}
What is the distance of this binary? 
Can you resolve the individual stars from the ground (1" resolution set by seeing) or from the Hubble Space Telescope (resolution $\sim\lambda/D$, where $D = 2.3-m$ is the diameter of the telescope and $\lambda$ is in the optical)?



%==============================================================
\section{}
(20 points) An X-ray source is detected near the plane of the Galaxy. 
This source may represent an X-ray binary consisting of an accreting compact object (e.g., neutron star) and a normal star as the companion.
The X-ray spectrum of the source indicates an interstellar absorbing gas column density $N_H = (5 \pm 0.5) \times 10^{22} cm^{-2}$ along the line of sight.
Within the position error circle of the source, a near-IR object is identified in the 2MASS catalog.
The measured J, H, and K band magnitudes of this object are
$12:66\pm0.03$, $11.77\pm0.03$, and $11.51\pm0.06$, respectively.
The obvious question is whether or not this near-IR object is a potential counterpart of the X-ray source.
To address this question, please do the following (aided by Ducati et al. 2001, ApJ, 558, 309):
\subsection{}
Draw an J-H vs. H-K diagram, marking the intrinsic color ranges of main sequence stars, giants, and super-giants, separately (Tables 3-5 in Ducati et al. 2001).
For each of these luminosity types, show how the color range would change with extinction (e.g., drawing a vector corresponding to $E(B-V)=1$) in this so-called color-color diagram, based on the Galactic extinction law (e.g., Table 3; Cardelli et al. 1989, ApJ, 345, 245).
\subsection{}
Assuming that the near-IR object is either a main sequence star or a giant star, estimate or constrain its extinction and compare it with the value inferred from the X-ray absorbing gas column density toward the X-ray source (using the $N_H/A_V = 1.8\times 10^{21} \mathrm{atoms} cm^{-2} mag^{-1}$).
Is the near-IR object likely to be the counterpart of the X-ray
source?
Similar methods may be used to probe the nature of other types of objects (e.g., galaxies, accounting for extinction and red-shift effects).



%==============================================================
\section{}
(10 points) Assuming that the star formation rate for the Galaxy is $6 M_\odot \si{\per\year}$ for the last 1 Gyr or so.
\subsection{} 
Please express and plot an appropriately-scaled differential Salpeter initial mass function (IMF) with an index of $x = 2.35$ for $M \geq 0.5M_\odot$ and $x = 1.3$ for
$0.1M_\odot\leq M < 0.5M_\odot$; the IMF should have a continuity at the break point.
\subsection{}
Further assuming that all stars between $8$ and $100M_\odot$ explode as core-collapsed supernovae, estimate the corresponding present supernova rate. 
How much difference does it make if the upper mass limit for supernovae is assumed to be infinite? (The observed rate for
core-collapsed supernovae in galaxies like our own is of the order of 2 to 3 per century.)



