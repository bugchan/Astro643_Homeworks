\section{}(10 points)
\textbf{Estimate the gravitational bounding energy of a white dwarf and compare this energy to that of a neutron star. Would the energy released from the nuclear burning of a 1.4 $M_\odot$ Carbon white dwarf unbound it? Could the energy unbound a neutron star
of a similar mass?}

The gravitational potential is given by
\begin{equation*}
    \Omega = \frac{3}{5-n}\frac{GM^2}{R}
\end{equation*}
where $n$ is the polytrope index. For both the neutron star and a white dwarf, the polytrope index is $n=3$  which corresponds to a relativistic degenerate gas. Given a mass of $1.4M_\odot$, the gravitational potential for a white dwarf and a neutron star will be 
\begin{equation*}
    \Omega_{WD}=\frac{3}{2}\frac{G(1.4M_\odot)^2}{R_{WD}}=\SI{1.22e+51}{\erg} \quad\text{and}\quad \Omega_{NS}=\frac{3}{2}\frac{G(1.4M_\odot)^2}{R_{NS}} =\SI{7.77e+55}{\erg} , \text{ respectively.}
\end{equation*}
considering that the white dwarf has $R_{WD}\sim R_\oplus$ and a neutron star has $R_{NS} \sim \SI{10}{\km}$, obtained from Fig.2.33 of the textbook.

The Q-value for the formation of ${}^{56}\mathrm{Ni}$ from ${}^{12}\mathrm{C}$ is $\SI{8.25e-5}{\erg}$. So energy released from the nuclear burning of carbon in a white dwarf is
\begin{equation*}
    E = Q\frac{M N_A}{A_C} = \SI{8.25e-5}{\erg}\frac{1.4M_\odot( \num{6.022e23}\text{ mol}^{-1})}{12 \text{ amu}} = \SI{1.15e+52}{\erg}
\end{equation*}
where $M$ is the total mass of star, $A_C$ is the atomic weight of the Carbon atom and $N_A$ is the Avogadro's number.

Note that the energy from nuclear burning is higher than the gravitational energy of the white dwarf so it is enough to unbound it but not enough to unbound a neutron star of similar mass.


%==============================================================
\newpage
\section{}(20 points)
\textbf{Consider the hypothetical evolution of a star of initial mass $M_0$.
The star's core grows in mass as a result of nuclear burning.
The nuclear processes release an amount of energy $Q$ per gram of burnt material. 
The star loses mass (by means of a stellar wind) at a rate proportional to its constant luminosity $L$, $\Dot{M} = \alpha L$.}
\subsection{}
\textbf{Find the mass of the core as a function of time, $M_c(t)$, assuming that $M_c(0) = 0$.}

The rate of change of the mass core is given by 
\begin{equation*}
    \frac{dM_c}{dt} = \frac{L}{Q}
\end{equation*}
where L is the luminosity and Q is the energy per gram released from nuclear burning. 

We can solve for the core mass by doing a separation of variables and using constant luminosity, we have that
\begin{equation*}
    \int_{M_c} dM_c = \int_t \frac{L}{Q}dt \qquad\rightarrow
    \qquad M_c = \frac{L}{Q}t + \mathrm{constant}
\end{equation*}

Using the initial condition of $M_c(t=0)=0$, we find that the constant $ = 0$, so that the core mass function is
\begin{align}
    M_c = \frac{L}{Q}t
\end{align}


\subsection{}
\textbf{Assuming that the initial mass of the envelope is $M_e(0) = M_0$, what is the core mass when the envelope mass vanishes?}

The mass of the envelope will have losses due to the nuclear burning sending materials to the core and due to the stellar winds, so 
\begin{align*}
    \frac{dM_e}{dt} &= -\Dot{M_c}-\Dot{M} = -\frac{L}{Q} - \alpha L\\
    dM_e &= \left(-\frac{L}{Q} -\alpha L\right)dt \\
    M_e &= \left(-\frac{L}{Q} -\alpha L\right)t + constant \qquad\rightarrow\qquad M_e(t=0) = constant = M_0 \\
    M_e &= \left(-\frac{L}{Q} -\alpha L\right)t + M_0
\end{align*}

To find the core mass when the envelope vanishes we first find the time it takes the envelope to vanish, so we solve for $t$,
\begin{equation*}
    \left(\frac{L}{Q} +\alpha L\right)t = M_0 \qquad\rightarrow\qquad t = \frac{M_0}{\frac{L}{Q} +\alpha L}
\end{equation*}

Then, the core mass when the envelope vanishes is
\begin{equation}
    M_c = \frac{L}{Q}\left(\frac{M_0}{\frac{L}{Q} +\alpha L}\right) = \frac{M_0}{1+Q\alpha}
    \label{eq:2coreMass}
\end{equation}


\subsection{}
\textbf{Calculate the upper limit of $M_0$, for which the star will become a white dwarf (considering $M_{Ch} = 1.4M_\odot$), given $Q = \SI{5e18}{\erg\per\g}$ (from turning solar composition
into carbon and oxygen) and $\alpha = \SI{1e-18}{g\per\erg}$.}

Given that the Chandrasekhar mass is the limit mass for a fully relativistic electron degenerate core, we can find an upper limit for $M_0$. Solving $M_0$ from Eq. \ref{eq:2coreMass}, we find that
\begin{equation*}
    M_0 = M_c (1+Q\alpha) = 1.4M_\odot (1 + (\SI{5e18}{\erg\per\g})(\SI{1e-18}{g\per\erg}))= 8.4 M_\odot,
\end{equation*}
which is in agreement to the largest mass at which the electron degeneracy of a CO core is high enough to prevent carbon ignition at solar metallicity.



%==============================================================
\newpage
\section{}(40 points)
\textbf{In the following assume the visual magnitude of the Sun as seen from the Earth is $m_{V,\odot} = -26.5$, and the observed color index of the Sun is $(B-V)_\odot = 0.60$.}
\subsection{}
\textbf{You observe a star with a spectrum that appears to be identical to that of the Sun.
The star has an observed visual magnitude $m_V = 14$. How far away is the star (in parsecs) assuming there is no dust along the line of sight?}

We can find the distance by using the distance modulus by
\begin{equation*}
    m_*-M_* = 5\log\left(\frac{d}{10pc}\right)=5\log(d)-5 \qquad\rightarrow\qquad \log(d) = \frac{m_*-M_*+5}{5}
\end{equation*}
where $m_*$ is the apparent magnitude, $M_*$ is the absolute magnitude and $d$ is the distance in parsecs.

Assuming that the absolute magnitude of the star is $M_{v,*} = M_{v,\odot}=5.072$, then the distance to the star is
\begin{equation}
    \log(d) = (14-5.072+5)/5 = 2.786 \qquad\rightarrow\qquad d = 10^{2.786}= \SI{610.4}{\parsec}
\end{equation}


\subsection{}
\textbf{Someone later observes that the color of the star is $B-V = 1.10$. 
Recalculate the distance assuming a standard extinction law with $R = 3.1$.}
Assuming the intrinsic color $(B-V)_{0,*} = (B-V)_{0,\odot}=0.60$, we can find that the color excess is
\begin{equation*}
    E(B-V) = (B-V)_* - (B-V)_{0,*} = 1.10 - 0.60 = 0.50
\end{equation*}
Then we know that $R_V = \frac{A_V}{E(B-V)}$, so we can find $A_V$
\begin{equation*}
    A_v = R_v*E(B-V) = 3.1(0.50) = 1.55
\end{equation*}
then the intrinsic magnitude is
\begin{equation*}
    m_{0,*} = m_{v,*} - A_v = 14 - 1.55 = 12.45
\end{equation*}
so the corrected distance will be
\begin{equation}
    \log(d) = (12.45-5.072+5)/5 = 2.476 \qquad\rightarrow\qquad d = 10^{2.476}= \SI{299}{\parsec}
\end{equation}


\subsection{}
\textbf{What would you expect the J magnitude of this star to be?
You will need to look up the intrinsic colors of a G2 V star somewhere, for example, Kenyon \& Hartmann (1995, ApJS 101, 117) and take into account any corrections for reddening at J.
If the J-magnitude turned out to be brighter than expected, what explanation might you put forward?}

From Cardelli (1989) we obtained that $A_J / A_V = 0.282$ so that the extinction in J band is 
\begin{equation}
    A_J = 0.282*A_V = 0.282(1.55) = 0.4371
\end{equation}
thus, the color excess is 
\begin{equation*}
    E(V-J) = A_V - A_J = 1.55 - 0.4371 = 1.113
\end{equation*}

Then, given that the intrinsic color of a G2 V star is $(V-J)_0=1.09$ obtained from Kenyon \& Hartmann (1995), we can find the expected J magnitude.
\begin{equation*}
    V-J = (V-J)_0 + E(V-J) = 1.09 + 1.113 = 2.203 \qquad\rightarrow\qquad J= V-2.203 = 14 - 2.203 = 11.797
\end{equation*}

If the J-magnitude turn out to be brighter than expected, then we have two options. One option is that there might be something else that is contributing to the brightness in that band giving a brighter magnitude than expected. The other option is that the extinction law is different than what we are assuming giving a difference between the expected and real magnitude.  


\subsection{}
\textbf{Oops. Now a new observation shows that in fact the star is a double-lined spectroscopic binary, with both components having identical spectral types.
The half-amplitude of the radial velocity curves is \SI{20}{\kilo\meter\per\second} for each component, and the period is 25 days.
Find the inclination of the orbit (0 degrees means the orbit is in the plane of the sky).
Is the system an eclipsing binary?}

The velocity is related to the inclination angle, $i$, and the period, $P$, by
\begin{equation}
    v_{obs} = \frac{2\pi a_1 \sin(i)}{P}, \quad\mathrm{where}\quad a_1 = \frac{am_2}{m_1+m_2}
\end{equation}
Using Kepler third law and using the fact that $m_1=m_2=m_\odot$, we can solve for the inclination angle by
\begin{align*}
    P^2 &= \frac{4\pi^2}{G(m_1+m_2)}a^3 \qquad\rightarrow\qquad a = \left[\frac{P^2G(m_1+m_2)}{4\pi^2}\right]^{1/3}\\
    a_1 &= \frac{m_2}{m_1+m_2}\left[\frac{P^2G(m_1+m_2)}{4\pi^2}\right]^{1/3} = \left[\frac{m_2^3P^2G}{4\pi^2(m_1+m_2)^2}\right]^{1/3} = \left[\frac{m_\odot^3P^2G}{4\pi^2(2m_\odot)^2}\right]^{1/3} = \left[\frac{m_\odot P^2G}{16\pi^2}\right]^{1/3}\\
    \sin(i) &= \frac{Pv_{obs}}{2\pi a_1} = \frac{Pv_{obs}}{2\pi}\left[\frac{16\pi^2}{m_\odot P^2G}\right]^{1/3} = v_{obs}\left[\frac{2P}{\pi m_\odot G}\right]^{1/3} = 0.436\\
    i &= 0.451\text{ rad} = 25.845^\circ
\end{align*}

The binary system is not eclipsing since the inclination angle is small.


\subsection{}
\textbf{What is the distance of this binary? 
Can you resolve the individual stars from the ground (1" resolution set by seeing) or from the Hubble Space Telescope (resolution $\sim\lambda/D$, where $D = \SI{2.3}{\m}$ is the diameter of the telescope and $\lambda$ is in the optical)?}

The separation between the stars is given by the semi-major axis, $a$, from Kepler's third law so that
\begin{equation*}
    a = \left[\frac{P^2G(2m_\odot)}{4\pi^2}\right]^{1/3} = \num{2.109e-01}\mathrm{AU} 
\end{equation*}

The angular separation in the sky of the binary system can be determined by using the small angle approximation as 
\begin{equation*}
    \theta\sim\frac{d}{D} = \frac{2.109e-01}{} = 
\end{equation*}
where $d$ is the distance between the stars in AU and $D$ is the distance to the binary system in pc.

The Hubble resolution is given by
\begin{equation*}
    \theta_{Hubble}\sim\frac{\lambda_V}{D} = \frac{\SI{551}{\nano\meter}}{\SI{2.3}{\meter}} = \num{2.396e-07}\text{ rad} = \num{4.941e-02}\text{ arcsec}
\end{equation*}



%==============================================================
\newpage
\section{}(20 points)
\textbf{An X-ray source is detected near the plane of the Galaxy. 
This source may represent an X-ray binary consisting of an accreting compact object (e.g., neutron star) and a normal star as the companion.
The X-ray spectrum of the source indicates an interstellar absorbing gas column density $N_H = (5 \pm 0.5) \times 10^{22} cm^{-2}$ along the line of sight.
Within the position error circle of the source, a near-IR object is identified in the 2MASS catalog.
The measured J, H, and K band magnitudes of this object are
$12:66\pm0.03$, $11.77\pm0.03$, and $11.51\pm0.06$, respectively.
The obvious question is whether or not this near-IR object is a potential counterpart of the X-ray source.
To address this question, please do the following (aided by Ducati et al. 2001, ApJ, 558, 309):}
\subsection{}
\textbf{Draw an J-H vs. H-K diagram, marking the intrinsic color ranges of main sequence stars, giants, and super-giants, separately (Tables 3-5 in Ducati et al. 2001).
For each of these luminosity types, show how the color range would change with extinction (e.g., drawing a vector corresponding to $E(B-V)=1$) in this so-called color-color diagram, based on the Galactic extinction law (e.g., Table 3; Cardelli et al. 1989, ApJ, 345, 245).}

To do the plot, we first needed to determine the colors by doing the following to all stars in the Ducati's Tables,
\begin{equation*}
    (J-H)_0 = (V-H)_0 - (V-J)_0\qquad\text{and }
    (H-K)_0 = (V-K)_0 - (V-H)_0
\end{equation*}

From the extinction law from Cardelli (1989), we can obtain the extinction on each band by
\begin{equation*}
    A_J = 0.282A_V,\quad A_H = 0.19A_V,\quad A_K = 0.114A_V
\end{equation*}
so the color excess are given by
\begin{align*}
    E(J-H) &= A_J - A_H = 0.282A_V - 0.19A_V = 0.092A_V\\
    E(H-K) &= A_H - A_K = 0.19A_V - 0.114A_V = 0.076A_V
\end{align*}

Given that the color excess $E(B-V)=1$ and $R_V=3.1$, then we can find that $A_v = R_V * E(B-V) = 3.1$. 
We can then use this extinction to determine our extinction vector which will be "described" as $E(H-K)\uveci + E(J-H)\uvecj$.

Figure \ref{fig:ColorColorDiagram} shows the resultant J-H vs H-K diagram at which the arrow shows how the color will change with extinction of $A_v=3.1$. 

\begin{figure}
    \centering
    %% Creator: Matplotlib, PGF backend
%%
%% To include the figure in your LaTeX document, write
%%   \input{<filename>.pgf}
%%
%% Make sure the required packages are loaded in your preamble
%%   \usepackage{pgf}
%%
%% Figures using additional raster images can only be included by \input if
%% they are in the same directory as the main LaTeX file. For loading figures
%% from other directories you can use the `import` package
%%   \usepackage{import}
%% and then include the figures with
%%   \import{<path to file>}{<filename>.pgf}
%%
%% Matplotlib used the following preamble
%%   \usepackage{fontspec}
%%
\begingroup%
\makeatletter%
\begin{pgfpicture}%
\pgfpathrectangle{\pgfpointorigin}{\pgfqpoint{6.780000in}{2.095135in}}%
\pgfusepath{use as bounding box, clip}%
\begin{pgfscope}%
\pgfsetbuttcap%
\pgfsetmiterjoin%
\definecolor{currentfill}{rgb}{1.000000,1.000000,1.000000}%
\pgfsetfillcolor{currentfill}%
\pgfsetlinewidth{0.000000pt}%
\definecolor{currentstroke}{rgb}{1.000000,1.000000,1.000000}%
\pgfsetstrokecolor{currentstroke}%
\pgfsetdash{}{0pt}%
\pgfpathmoveto{\pgfqpoint{0.000000in}{0.000000in}}%
\pgfpathlineto{\pgfqpoint{6.780000in}{0.000000in}}%
\pgfpathlineto{\pgfqpoint{6.780000in}{2.095135in}}%
\pgfpathlineto{\pgfqpoint{0.000000in}{2.095135in}}%
\pgfpathclose%
\pgfusepath{fill}%
\end{pgfscope}%
\begin{pgfscope}%
\pgfsetbuttcap%
\pgfsetmiterjoin%
\definecolor{currentfill}{rgb}{1.000000,1.000000,1.000000}%
\pgfsetfillcolor{currentfill}%
\pgfsetlinewidth{0.000000pt}%
\definecolor{currentstroke}{rgb}{0.000000,0.000000,0.000000}%
\pgfsetstrokecolor{currentstroke}%
\pgfsetstrokeopacity{0.000000}%
\pgfsetdash{}{0pt}%
\pgfpathmoveto{\pgfqpoint{0.847500in}{0.261892in}}%
\pgfpathlineto{\pgfqpoint{2.489531in}{0.261892in}}%
\pgfpathlineto{\pgfqpoint{2.489531in}{1.843719in}}%
\pgfpathlineto{\pgfqpoint{0.847500in}{1.843719in}}%
\pgfpathclose%
\pgfusepath{fill}%
\end{pgfscope}%
\begin{pgfscope}%
\pgfpathrectangle{\pgfqpoint{0.847500in}{0.261892in}}{\pgfqpoint{1.642031in}{1.581827in}}%
\pgfusepath{clip}%
\pgfsetrectcap%
\pgfsetroundjoin%
\pgfsetlinewidth{0.281050pt}%
\definecolor{currentstroke}{rgb}{0.690196,0.690196,0.690196}%
\pgfsetstrokecolor{currentstroke}%
\pgfsetdash{}{0pt}%
\pgfpathmoveto{\pgfqpoint{0.944090in}{0.261892in}}%
\pgfpathlineto{\pgfqpoint{0.944090in}{1.843719in}}%
\pgfusepath{stroke}%
\end{pgfscope}%
\begin{pgfscope}%
\pgfsetbuttcap%
\pgfsetroundjoin%
\definecolor{currentfill}{rgb}{0.000000,0.000000,0.000000}%
\pgfsetfillcolor{currentfill}%
\pgfsetlinewidth{0.803000pt}%
\definecolor{currentstroke}{rgb}{0.000000,0.000000,0.000000}%
\pgfsetstrokecolor{currentstroke}%
\pgfsetdash{}{0pt}%
\pgfsys@defobject{currentmarker}{\pgfqpoint{0.000000in}{-0.048611in}}{\pgfqpoint{0.000000in}{0.000000in}}{%
\pgfpathmoveto{\pgfqpoint{0.000000in}{0.000000in}}%
\pgfpathlineto{\pgfqpoint{0.000000in}{-0.048611in}}%
\pgfusepath{stroke,fill}%
}%
\begin{pgfscope}%
\pgfsys@transformshift{0.944090in}{0.261892in}%
\pgfsys@useobject{currentmarker}{}%
\end{pgfscope}%
\end{pgfscope}%
\begin{pgfscope}%
\definecolor{textcolor}{rgb}{0.000000,0.000000,0.000000}%
\pgfsetstrokecolor{textcolor}%
\pgfsetfillcolor{textcolor}%
\pgftext[x=0.944090in,y=0.164670in,,top]{\color{textcolor}\sffamily\fontsize{8.000000}{9.600000}\selectfont −1.0}%
\end{pgfscope}%
\begin{pgfscope}%
\pgfpathrectangle{\pgfqpoint{0.847500in}{0.261892in}}{\pgfqpoint{1.642031in}{1.581827in}}%
\pgfusepath{clip}%
\pgfsetrectcap%
\pgfsetroundjoin%
\pgfsetlinewidth{0.281050pt}%
\definecolor{currentstroke}{rgb}{0.690196,0.690196,0.690196}%
\pgfsetstrokecolor{currentstroke}%
\pgfsetdash{}{0pt}%
\pgfpathmoveto{\pgfqpoint{1.427040in}{0.261892in}}%
\pgfpathlineto{\pgfqpoint{1.427040in}{1.843719in}}%
\pgfusepath{stroke}%
\end{pgfscope}%
\begin{pgfscope}%
\pgfsetbuttcap%
\pgfsetroundjoin%
\definecolor{currentfill}{rgb}{0.000000,0.000000,0.000000}%
\pgfsetfillcolor{currentfill}%
\pgfsetlinewidth{0.803000pt}%
\definecolor{currentstroke}{rgb}{0.000000,0.000000,0.000000}%
\pgfsetstrokecolor{currentstroke}%
\pgfsetdash{}{0pt}%
\pgfsys@defobject{currentmarker}{\pgfqpoint{0.000000in}{-0.048611in}}{\pgfqpoint{0.000000in}{0.000000in}}{%
\pgfpathmoveto{\pgfqpoint{0.000000in}{0.000000in}}%
\pgfpathlineto{\pgfqpoint{0.000000in}{-0.048611in}}%
\pgfusepath{stroke,fill}%
}%
\begin{pgfscope}%
\pgfsys@transformshift{1.427040in}{0.261892in}%
\pgfsys@useobject{currentmarker}{}%
\end{pgfscope}%
\end{pgfscope}%
\begin{pgfscope}%
\definecolor{textcolor}{rgb}{0.000000,0.000000,0.000000}%
\pgfsetstrokecolor{textcolor}%
\pgfsetfillcolor{textcolor}%
\pgftext[x=1.427040in,y=0.164670in,,top]{\color{textcolor}\sffamily\fontsize{8.000000}{9.600000}\selectfont −0.5}%
\end{pgfscope}%
\begin{pgfscope}%
\pgfpathrectangle{\pgfqpoint{0.847500in}{0.261892in}}{\pgfqpoint{1.642031in}{1.581827in}}%
\pgfusepath{clip}%
\pgfsetrectcap%
\pgfsetroundjoin%
\pgfsetlinewidth{0.281050pt}%
\definecolor{currentstroke}{rgb}{0.690196,0.690196,0.690196}%
\pgfsetstrokecolor{currentstroke}%
\pgfsetdash{}{0pt}%
\pgfpathmoveto{\pgfqpoint{1.909991in}{0.261892in}}%
\pgfpathlineto{\pgfqpoint{1.909991in}{1.843719in}}%
\pgfusepath{stroke}%
\end{pgfscope}%
\begin{pgfscope}%
\pgfsetbuttcap%
\pgfsetroundjoin%
\definecolor{currentfill}{rgb}{0.000000,0.000000,0.000000}%
\pgfsetfillcolor{currentfill}%
\pgfsetlinewidth{0.803000pt}%
\definecolor{currentstroke}{rgb}{0.000000,0.000000,0.000000}%
\pgfsetstrokecolor{currentstroke}%
\pgfsetdash{}{0pt}%
\pgfsys@defobject{currentmarker}{\pgfqpoint{0.000000in}{-0.048611in}}{\pgfqpoint{0.000000in}{0.000000in}}{%
\pgfpathmoveto{\pgfqpoint{0.000000in}{0.000000in}}%
\pgfpathlineto{\pgfqpoint{0.000000in}{-0.048611in}}%
\pgfusepath{stroke,fill}%
}%
\begin{pgfscope}%
\pgfsys@transformshift{1.909991in}{0.261892in}%
\pgfsys@useobject{currentmarker}{}%
\end{pgfscope}%
\end{pgfscope}%
\begin{pgfscope}%
\definecolor{textcolor}{rgb}{0.000000,0.000000,0.000000}%
\pgfsetstrokecolor{textcolor}%
\pgfsetfillcolor{textcolor}%
\pgftext[x=1.909991in,y=0.164670in,,top]{\color{textcolor}\sffamily\fontsize{8.000000}{9.600000}\selectfont 0.0}%
\end{pgfscope}%
\begin{pgfscope}%
\pgfpathrectangle{\pgfqpoint{0.847500in}{0.261892in}}{\pgfqpoint{1.642031in}{1.581827in}}%
\pgfusepath{clip}%
\pgfsetrectcap%
\pgfsetroundjoin%
\pgfsetlinewidth{0.281050pt}%
\definecolor{currentstroke}{rgb}{0.690196,0.690196,0.690196}%
\pgfsetstrokecolor{currentstroke}%
\pgfsetdash{}{0pt}%
\pgfpathmoveto{\pgfqpoint{2.392941in}{0.261892in}}%
\pgfpathlineto{\pgfqpoint{2.392941in}{1.843719in}}%
\pgfusepath{stroke}%
\end{pgfscope}%
\begin{pgfscope}%
\pgfsetbuttcap%
\pgfsetroundjoin%
\definecolor{currentfill}{rgb}{0.000000,0.000000,0.000000}%
\pgfsetfillcolor{currentfill}%
\pgfsetlinewidth{0.803000pt}%
\definecolor{currentstroke}{rgb}{0.000000,0.000000,0.000000}%
\pgfsetstrokecolor{currentstroke}%
\pgfsetdash{}{0pt}%
\pgfsys@defobject{currentmarker}{\pgfqpoint{0.000000in}{-0.048611in}}{\pgfqpoint{0.000000in}{0.000000in}}{%
\pgfpathmoveto{\pgfqpoint{0.000000in}{0.000000in}}%
\pgfpathlineto{\pgfqpoint{0.000000in}{-0.048611in}}%
\pgfusepath{stroke,fill}%
}%
\begin{pgfscope}%
\pgfsys@transformshift{2.392941in}{0.261892in}%
\pgfsys@useobject{currentmarker}{}%
\end{pgfscope}%
\end{pgfscope}%
\begin{pgfscope}%
\definecolor{textcolor}{rgb}{0.000000,0.000000,0.000000}%
\pgfsetstrokecolor{textcolor}%
\pgfsetfillcolor{textcolor}%
\pgftext[x=2.392941in,y=0.164670in,,top]{\color{textcolor}\sffamily\fontsize{8.000000}{9.600000}\selectfont 0.5}%
\end{pgfscope}%
\begin{pgfscope}%
\definecolor{textcolor}{rgb}{0.000000,0.000000,0.000000}%
\pgfsetstrokecolor{textcolor}%
\pgfsetfillcolor{textcolor}%
\pgftext[x=1.668516in,y=0.010448in,,top]{\color{textcolor}\sffamily\fontsize{8.000000}{9.600000}\selectfont \(\displaystyle (H-K)_0\)}%
\end{pgfscope}%
\begin{pgfscope}%
\pgfpathrectangle{\pgfqpoint{0.847500in}{0.261892in}}{\pgfqpoint{1.642031in}{1.581827in}}%
\pgfusepath{clip}%
\pgfsetrectcap%
\pgfsetroundjoin%
\pgfsetlinewidth{0.281050pt}%
\definecolor{currentstroke}{rgb}{0.690196,0.690196,0.690196}%
\pgfsetstrokecolor{currentstroke}%
\pgfsetdash{}{0pt}%
\pgfpathmoveto{\pgfqpoint{0.847500in}{1.843719in}}%
\pgfpathlineto{\pgfqpoint{2.489531in}{1.843719in}}%
\pgfusepath{stroke}%
\end{pgfscope}%
\begin{pgfscope}%
\pgfsetbuttcap%
\pgfsetroundjoin%
\definecolor{currentfill}{rgb}{0.000000,0.000000,0.000000}%
\pgfsetfillcolor{currentfill}%
\pgfsetlinewidth{0.803000pt}%
\definecolor{currentstroke}{rgb}{0.000000,0.000000,0.000000}%
\pgfsetstrokecolor{currentstroke}%
\pgfsetdash{}{0pt}%
\pgfsys@defobject{currentmarker}{\pgfqpoint{-0.048611in}{0.000000in}}{\pgfqpoint{0.000000in}{0.000000in}}{%
\pgfpathmoveto{\pgfqpoint{0.000000in}{0.000000in}}%
\pgfpathlineto{\pgfqpoint{-0.048611in}{0.000000in}}%
\pgfusepath{stroke,fill}%
}%
\begin{pgfscope}%
\pgfsys@transformshift{0.847500in}{1.843719in}%
\pgfsys@useobject{currentmarker}{}%
\end{pgfscope}%
\end{pgfscope}%
\begin{pgfscope}%
\definecolor{textcolor}{rgb}{0.000000,0.000000,0.000000}%
\pgfsetstrokecolor{textcolor}%
\pgfsetfillcolor{textcolor}%
\pgftext[x=0.448722in,y=1.805163in,left,base]{\color{textcolor}\sffamily\fontsize{8.000000}{9.600000}\selectfont −0.25}%
\end{pgfscope}%
\begin{pgfscope}%
\pgfpathrectangle{\pgfqpoint{0.847500in}{0.261892in}}{\pgfqpoint{1.642031in}{1.581827in}}%
\pgfusepath{clip}%
\pgfsetrectcap%
\pgfsetroundjoin%
\pgfsetlinewidth{0.281050pt}%
\definecolor{currentstroke}{rgb}{0.690196,0.690196,0.690196}%
\pgfsetstrokecolor{currentstroke}%
\pgfsetdash{}{0pt}%
\pgfpathmoveto{\pgfqpoint{0.847500in}{1.604048in}}%
\pgfpathlineto{\pgfqpoint{2.489531in}{1.604048in}}%
\pgfusepath{stroke}%
\end{pgfscope}%
\begin{pgfscope}%
\pgfsetbuttcap%
\pgfsetroundjoin%
\definecolor{currentfill}{rgb}{0.000000,0.000000,0.000000}%
\pgfsetfillcolor{currentfill}%
\pgfsetlinewidth{0.803000pt}%
\definecolor{currentstroke}{rgb}{0.000000,0.000000,0.000000}%
\pgfsetstrokecolor{currentstroke}%
\pgfsetdash{}{0pt}%
\pgfsys@defobject{currentmarker}{\pgfqpoint{-0.048611in}{0.000000in}}{\pgfqpoint{0.000000in}{0.000000in}}{%
\pgfpathmoveto{\pgfqpoint{0.000000in}{0.000000in}}%
\pgfpathlineto{\pgfqpoint{-0.048611in}{0.000000in}}%
\pgfusepath{stroke,fill}%
}%
\begin{pgfscope}%
\pgfsys@transformshift{0.847500in}{1.604048in}%
\pgfsys@useobject{currentmarker}{}%
\end{pgfscope}%
\end{pgfscope}%
\begin{pgfscope}%
\definecolor{textcolor}{rgb}{0.000000,0.000000,0.000000}%
\pgfsetstrokecolor{textcolor}%
\pgfsetfillcolor{textcolor}%
\pgftext[x=0.540500in,y=1.565493in,left,base]{\color{textcolor}\sffamily\fontsize{8.000000}{9.600000}\selectfont 0.00}%
\end{pgfscope}%
\begin{pgfscope}%
\pgfpathrectangle{\pgfqpoint{0.847500in}{0.261892in}}{\pgfqpoint{1.642031in}{1.581827in}}%
\pgfusepath{clip}%
\pgfsetrectcap%
\pgfsetroundjoin%
\pgfsetlinewidth{0.281050pt}%
\definecolor{currentstroke}{rgb}{0.690196,0.690196,0.690196}%
\pgfsetstrokecolor{currentstroke}%
\pgfsetdash{}{0pt}%
\pgfpathmoveto{\pgfqpoint{0.847500in}{1.364377in}}%
\pgfpathlineto{\pgfqpoint{2.489531in}{1.364377in}}%
\pgfusepath{stroke}%
\end{pgfscope}%
\begin{pgfscope}%
\pgfsetbuttcap%
\pgfsetroundjoin%
\definecolor{currentfill}{rgb}{0.000000,0.000000,0.000000}%
\pgfsetfillcolor{currentfill}%
\pgfsetlinewidth{0.803000pt}%
\definecolor{currentstroke}{rgb}{0.000000,0.000000,0.000000}%
\pgfsetstrokecolor{currentstroke}%
\pgfsetdash{}{0pt}%
\pgfsys@defobject{currentmarker}{\pgfqpoint{-0.048611in}{0.000000in}}{\pgfqpoint{0.000000in}{0.000000in}}{%
\pgfpathmoveto{\pgfqpoint{0.000000in}{0.000000in}}%
\pgfpathlineto{\pgfqpoint{-0.048611in}{0.000000in}}%
\pgfusepath{stroke,fill}%
}%
\begin{pgfscope}%
\pgfsys@transformshift{0.847500in}{1.364377in}%
\pgfsys@useobject{currentmarker}{}%
\end{pgfscope}%
\end{pgfscope}%
\begin{pgfscope}%
\definecolor{textcolor}{rgb}{0.000000,0.000000,0.000000}%
\pgfsetstrokecolor{textcolor}%
\pgfsetfillcolor{textcolor}%
\pgftext[x=0.540500in,y=1.325822in,left,base]{\color{textcolor}\sffamily\fontsize{8.000000}{9.600000}\selectfont 0.25}%
\end{pgfscope}%
\begin{pgfscope}%
\pgfpathrectangle{\pgfqpoint{0.847500in}{0.261892in}}{\pgfqpoint{1.642031in}{1.581827in}}%
\pgfusepath{clip}%
\pgfsetrectcap%
\pgfsetroundjoin%
\pgfsetlinewidth{0.281050pt}%
\definecolor{currentstroke}{rgb}{0.690196,0.690196,0.690196}%
\pgfsetstrokecolor{currentstroke}%
\pgfsetdash{}{0pt}%
\pgfpathmoveto{\pgfqpoint{0.847500in}{1.124707in}}%
\pgfpathlineto{\pgfqpoint{2.489531in}{1.124707in}}%
\pgfusepath{stroke}%
\end{pgfscope}%
\begin{pgfscope}%
\pgfsetbuttcap%
\pgfsetroundjoin%
\definecolor{currentfill}{rgb}{0.000000,0.000000,0.000000}%
\pgfsetfillcolor{currentfill}%
\pgfsetlinewidth{0.803000pt}%
\definecolor{currentstroke}{rgb}{0.000000,0.000000,0.000000}%
\pgfsetstrokecolor{currentstroke}%
\pgfsetdash{}{0pt}%
\pgfsys@defobject{currentmarker}{\pgfqpoint{-0.048611in}{0.000000in}}{\pgfqpoint{0.000000in}{0.000000in}}{%
\pgfpathmoveto{\pgfqpoint{0.000000in}{0.000000in}}%
\pgfpathlineto{\pgfqpoint{-0.048611in}{0.000000in}}%
\pgfusepath{stroke,fill}%
}%
\begin{pgfscope}%
\pgfsys@transformshift{0.847500in}{1.124707in}%
\pgfsys@useobject{currentmarker}{}%
\end{pgfscope}%
\end{pgfscope}%
\begin{pgfscope}%
\definecolor{textcolor}{rgb}{0.000000,0.000000,0.000000}%
\pgfsetstrokecolor{textcolor}%
\pgfsetfillcolor{textcolor}%
\pgftext[x=0.540500in,y=1.086151in,left,base]{\color{textcolor}\sffamily\fontsize{8.000000}{9.600000}\selectfont 0.50}%
\end{pgfscope}%
\begin{pgfscope}%
\pgfpathrectangle{\pgfqpoint{0.847500in}{0.261892in}}{\pgfqpoint{1.642031in}{1.581827in}}%
\pgfusepath{clip}%
\pgfsetrectcap%
\pgfsetroundjoin%
\pgfsetlinewidth{0.281050pt}%
\definecolor{currentstroke}{rgb}{0.690196,0.690196,0.690196}%
\pgfsetstrokecolor{currentstroke}%
\pgfsetdash{}{0pt}%
\pgfpathmoveto{\pgfqpoint{0.847500in}{0.885036in}}%
\pgfpathlineto{\pgfqpoint{2.489531in}{0.885036in}}%
\pgfusepath{stroke}%
\end{pgfscope}%
\begin{pgfscope}%
\pgfsetbuttcap%
\pgfsetroundjoin%
\definecolor{currentfill}{rgb}{0.000000,0.000000,0.000000}%
\pgfsetfillcolor{currentfill}%
\pgfsetlinewidth{0.803000pt}%
\definecolor{currentstroke}{rgb}{0.000000,0.000000,0.000000}%
\pgfsetstrokecolor{currentstroke}%
\pgfsetdash{}{0pt}%
\pgfsys@defobject{currentmarker}{\pgfqpoint{-0.048611in}{0.000000in}}{\pgfqpoint{0.000000in}{0.000000in}}{%
\pgfpathmoveto{\pgfqpoint{0.000000in}{0.000000in}}%
\pgfpathlineto{\pgfqpoint{-0.048611in}{0.000000in}}%
\pgfusepath{stroke,fill}%
}%
\begin{pgfscope}%
\pgfsys@transformshift{0.847500in}{0.885036in}%
\pgfsys@useobject{currentmarker}{}%
\end{pgfscope}%
\end{pgfscope}%
\begin{pgfscope}%
\definecolor{textcolor}{rgb}{0.000000,0.000000,0.000000}%
\pgfsetstrokecolor{textcolor}%
\pgfsetfillcolor{textcolor}%
\pgftext[x=0.540500in,y=0.846480in,left,base]{\color{textcolor}\sffamily\fontsize{8.000000}{9.600000}\selectfont 0.75}%
\end{pgfscope}%
\begin{pgfscope}%
\pgfpathrectangle{\pgfqpoint{0.847500in}{0.261892in}}{\pgfqpoint{1.642031in}{1.581827in}}%
\pgfusepath{clip}%
\pgfsetrectcap%
\pgfsetroundjoin%
\pgfsetlinewidth{0.281050pt}%
\definecolor{currentstroke}{rgb}{0.690196,0.690196,0.690196}%
\pgfsetstrokecolor{currentstroke}%
\pgfsetdash{}{0pt}%
\pgfpathmoveto{\pgfqpoint{0.847500in}{0.645365in}}%
\pgfpathlineto{\pgfqpoint{2.489531in}{0.645365in}}%
\pgfusepath{stroke}%
\end{pgfscope}%
\begin{pgfscope}%
\pgfsetbuttcap%
\pgfsetroundjoin%
\definecolor{currentfill}{rgb}{0.000000,0.000000,0.000000}%
\pgfsetfillcolor{currentfill}%
\pgfsetlinewidth{0.803000pt}%
\definecolor{currentstroke}{rgb}{0.000000,0.000000,0.000000}%
\pgfsetstrokecolor{currentstroke}%
\pgfsetdash{}{0pt}%
\pgfsys@defobject{currentmarker}{\pgfqpoint{-0.048611in}{0.000000in}}{\pgfqpoint{0.000000in}{0.000000in}}{%
\pgfpathmoveto{\pgfqpoint{0.000000in}{0.000000in}}%
\pgfpathlineto{\pgfqpoint{-0.048611in}{0.000000in}}%
\pgfusepath{stroke,fill}%
}%
\begin{pgfscope}%
\pgfsys@transformshift{0.847500in}{0.645365in}%
\pgfsys@useobject{currentmarker}{}%
\end{pgfscope}%
\end{pgfscope}%
\begin{pgfscope}%
\definecolor{textcolor}{rgb}{0.000000,0.000000,0.000000}%
\pgfsetstrokecolor{textcolor}%
\pgfsetfillcolor{textcolor}%
\pgftext[x=0.540500in,y=0.606810in,left,base]{\color{textcolor}\sffamily\fontsize{8.000000}{9.600000}\selectfont 1.00}%
\end{pgfscope}%
\begin{pgfscope}%
\pgfpathrectangle{\pgfqpoint{0.847500in}{0.261892in}}{\pgfqpoint{1.642031in}{1.581827in}}%
\pgfusepath{clip}%
\pgfsetrectcap%
\pgfsetroundjoin%
\pgfsetlinewidth{0.281050pt}%
\definecolor{currentstroke}{rgb}{0.690196,0.690196,0.690196}%
\pgfsetstrokecolor{currentstroke}%
\pgfsetdash{}{0pt}%
\pgfpathmoveto{\pgfqpoint{0.847500in}{0.405694in}}%
\pgfpathlineto{\pgfqpoint{2.489531in}{0.405694in}}%
\pgfusepath{stroke}%
\end{pgfscope}%
\begin{pgfscope}%
\pgfsetbuttcap%
\pgfsetroundjoin%
\definecolor{currentfill}{rgb}{0.000000,0.000000,0.000000}%
\pgfsetfillcolor{currentfill}%
\pgfsetlinewidth{0.803000pt}%
\definecolor{currentstroke}{rgb}{0.000000,0.000000,0.000000}%
\pgfsetstrokecolor{currentstroke}%
\pgfsetdash{}{0pt}%
\pgfsys@defobject{currentmarker}{\pgfqpoint{-0.048611in}{0.000000in}}{\pgfqpoint{0.000000in}{0.000000in}}{%
\pgfpathmoveto{\pgfqpoint{0.000000in}{0.000000in}}%
\pgfpathlineto{\pgfqpoint{-0.048611in}{0.000000in}}%
\pgfusepath{stroke,fill}%
}%
\begin{pgfscope}%
\pgfsys@transformshift{0.847500in}{0.405694in}%
\pgfsys@useobject{currentmarker}{}%
\end{pgfscope}%
\end{pgfscope}%
\begin{pgfscope}%
\definecolor{textcolor}{rgb}{0.000000,0.000000,0.000000}%
\pgfsetstrokecolor{textcolor}%
\pgfsetfillcolor{textcolor}%
\pgftext[x=0.540500in,y=0.367139in,left,base]{\color{textcolor}\sffamily\fontsize{8.000000}{9.600000}\selectfont 1.25}%
\end{pgfscope}%
\begin{pgfscope}%
\definecolor{textcolor}{rgb}{0.000000,0.000000,0.000000}%
\pgfsetstrokecolor{textcolor}%
\pgfsetfillcolor{textcolor}%
\pgftext[x=0.393167in,y=1.052805in,,bottom,rotate=90.000000]{\color{textcolor}\sffamily\fontsize{8.000000}{9.600000}\selectfont \(\displaystyle (J-H)_0\)}%
\end{pgfscope}%
\begin{pgfscope}%
\pgfpathrectangle{\pgfqpoint{0.847500in}{0.261892in}}{\pgfqpoint{1.642031in}{1.581827in}}%
\pgfusepath{clip}%
\pgfsetbuttcap%
\pgfsetmiterjoin%
\definecolor{currentfill}{rgb}{0.000000,0.000000,0.000000}%
\pgfsetfillcolor{currentfill}%
\pgfsetlinewidth{1.003750pt}%
\definecolor{currentstroke}{rgb}{0.000000,0.000000,0.000000}%
\pgfsetstrokecolor{currentstroke}%
\pgfsetdash{}{0pt}%
\pgfpathmoveto{\pgfqpoint{2.473465in}{0.795857in}}%
\pgfpathlineto{\pgfqpoint{2.445944in}{0.866554in}}%
\pgfpathlineto{\pgfqpoint{2.427700in}{0.851596in}}%
\pgfpathlineto{\pgfqpoint{2.200133in}{1.125012in}}%
\pgfpathlineto{\pgfqpoint{2.199389in}{1.124401in}}%
\pgfpathlineto{\pgfqpoint{2.426955in}{0.850985in}}%
\pgfpathlineto{\pgfqpoint{2.408710in}{0.836026in}}%
\pgfpathclose%
\pgfusepath{stroke,fill}%
\end{pgfscope}%
\begin{pgfscope}%
\pgfpathrectangle{\pgfqpoint{0.847500in}{0.261892in}}{\pgfqpoint{1.642031in}{1.581827in}}%
\pgfusepath{clip}%
\pgfsetbuttcap%
\pgfsetmiterjoin%
\definecolor{currentfill}{rgb}{0.000000,0.000000,0.000000}%
\pgfsetfillcolor{currentfill}%
\pgfsetlinewidth{1.003750pt}%
\definecolor{currentstroke}{rgb}{0.000000,0.000000,0.000000}%
\pgfsetstrokecolor{currentstroke}%
\pgfsetdash{}{0pt}%
\pgfpathmoveto{\pgfqpoint{2.127988in}{0.790633in}}%
\pgfpathlineto{\pgfqpoint{2.138997in}{0.762354in}}%
\pgfpathlineto{\pgfqpoint{2.146071in}{0.768155in}}%
\pgfpathlineto{\pgfqpoint{2.160753in}{0.750515in}}%
\pgfpathlineto{\pgfqpoint{2.161497in}{0.751126in}}%
\pgfpathlineto{\pgfqpoint{2.146816in}{0.768765in}}%
\pgfpathlineto{\pgfqpoint{2.153890in}{0.774566in}}%
\pgfpathclose%
\pgfusepath{stroke,fill}%
\end{pgfscope}%
\begin{pgfscope}%
\pgfpathrectangle{\pgfqpoint{0.847500in}{0.261892in}}{\pgfqpoint{1.642031in}{1.581827in}}%
\pgfusepath{clip}%
\pgfsetrectcap%
\pgfsetroundjoin%
\pgfsetlinewidth{0.401500pt}%
\definecolor{currentstroke}{rgb}{0.121569,0.466667,0.705882}%
\pgfsetstrokecolor{currentstroke}%
\pgfsetdash{}{0pt}%
\pgfpathmoveto{\pgfqpoint{1.861696in}{1.719090in}}%
\pgfpathlineto{\pgfqpoint{1.852037in}{1.719090in}}%
\pgfpathlineto{\pgfqpoint{1.832719in}{1.719090in}}%
\pgfpathlineto{\pgfqpoint{1.823060in}{1.719090in}}%
\pgfpathlineto{\pgfqpoint{1.813401in}{1.719090in}}%
\pgfpathlineto{\pgfqpoint{1.813401in}{1.719090in}}%
\pgfpathlineto{\pgfqpoint{1.813401in}{1.719090in}}%
\pgfpathlineto{\pgfqpoint{1.813401in}{1.719090in}}%
\pgfpathlineto{\pgfqpoint{1.823060in}{1.719090in}}%
\pgfpathlineto{\pgfqpoint{1.823060in}{1.709503in}}%
\pgfpathlineto{\pgfqpoint{1.823060in}{1.709503in}}%
\pgfpathlineto{\pgfqpoint{1.842378in}{1.709503in}}%
\pgfpathlineto{\pgfqpoint{1.852037in}{1.699917in}}%
\pgfpathlineto{\pgfqpoint{1.852037in}{1.690330in}}%
\pgfpathlineto{\pgfqpoint{1.871355in}{1.690330in}}%
\pgfpathlineto{\pgfqpoint{1.881014in}{1.690330in}}%
\pgfpathlineto{\pgfqpoint{1.919650in}{1.680743in}}%
\pgfpathlineto{\pgfqpoint{1.938968in}{1.671156in}}%
\pgfpathlineto{\pgfqpoint{1.929309in}{1.632809in}}%
\pgfpathlineto{\pgfqpoint{1.919650in}{1.613635in}}%
\pgfpathlineto{\pgfqpoint{1.900332in}{1.575288in}}%
\pgfpathlineto{\pgfqpoint{1.890673in}{1.556114in}}%
\pgfpathlineto{\pgfqpoint{1.881014in}{1.527354in}}%
\pgfpathlineto{\pgfqpoint{1.871355in}{1.508180in}}%
\pgfpathlineto{\pgfqpoint{1.823060in}{1.441072in}}%
\pgfpathlineto{\pgfqpoint{1.861696in}{1.469833in}}%
\pgfpathlineto{\pgfqpoint{1.852037in}{1.441072in}}%
\pgfpathlineto{\pgfqpoint{1.861696in}{1.431485in}}%
\pgfpathlineto{\pgfqpoint{1.861696in}{1.412312in}}%
\pgfpathlineto{\pgfqpoint{1.852037in}{1.402725in}}%
\pgfpathlineto{\pgfqpoint{1.861696in}{1.383551in}}%
\pgfpathlineto{\pgfqpoint{1.871355in}{1.354791in}}%
\pgfpathlineto{\pgfqpoint{1.881014in}{1.345204in}}%
\pgfpathlineto{\pgfqpoint{1.900332in}{1.335617in}}%
\pgfpathlineto{\pgfqpoint{1.919650in}{1.335617in}}%
\pgfpathlineto{\pgfqpoint{1.919650in}{1.326030in}}%
\pgfpathlineto{\pgfqpoint{1.919650in}{1.326030in}}%
\pgfpathlineto{\pgfqpoint{1.938968in}{1.316443in}}%
\pgfpathlineto{\pgfqpoint{1.977604in}{1.287683in}}%
\pgfpathlineto{\pgfqpoint{1.996922in}{1.278096in}}%
\pgfpathlineto{\pgfqpoint{2.025899in}{1.278096in}}%
\pgfpathlineto{\pgfqpoint{2.045217in}{1.258922in}}%
\pgfpathlineto{\pgfqpoint{2.064535in}{1.239749in}}%
\pgfpathlineto{\pgfqpoint{2.103171in}{1.201401in}}%
\pgfpathlineto{\pgfqpoint{2.132148in}{1.143880in}}%
\pgfpathlineto{\pgfqpoint{2.151466in}{1.076773in}}%
\pgfpathlineto{\pgfqpoint{2.151466in}{1.019252in}}%
\pgfpathlineto{\pgfqpoint{2.151466in}{0.952144in}}%
\pgfpathlineto{\pgfqpoint{2.151466in}{0.894623in}}%
\pgfpathlineto{\pgfqpoint{2.141807in}{0.827515in}}%
\pgfusepath{stroke}%
\end{pgfscope}%
\begin{pgfscope}%
\pgfpathrectangle{\pgfqpoint{0.847500in}{0.261892in}}{\pgfqpoint{1.642031in}{1.581827in}}%
\pgfusepath{clip}%
\pgfsetbuttcap%
\pgfsetroundjoin%
\definecolor{currentfill}{rgb}{0.121569,0.466667,0.705882}%
\pgfsetfillcolor{currentfill}%
\pgfsetlinewidth{1.003750pt}%
\definecolor{currentstroke}{rgb}{0.121569,0.466667,0.705882}%
\pgfsetstrokecolor{currentstroke}%
\pgfsetdash{}{0pt}%
\pgfsys@defobject{currentmarker}{\pgfqpoint{-0.006944in}{-0.006944in}}{\pgfqpoint{0.006944in}{0.006944in}}{%
\pgfpathmoveto{\pgfqpoint{0.000000in}{-0.006944in}}%
\pgfpathcurveto{\pgfqpoint{0.001842in}{-0.006944in}}{\pgfqpoint{0.003608in}{-0.006213in}}{\pgfqpoint{0.004910in}{-0.004910in}}%
\pgfpathcurveto{\pgfqpoint{0.006213in}{-0.003608in}}{\pgfqpoint{0.006944in}{-0.001842in}}{\pgfqpoint{0.006944in}{0.000000in}}%
\pgfpathcurveto{\pgfqpoint{0.006944in}{0.001842in}}{\pgfqpoint{0.006213in}{0.003608in}}{\pgfqpoint{0.004910in}{0.004910in}}%
\pgfpathcurveto{\pgfqpoint{0.003608in}{0.006213in}}{\pgfqpoint{0.001842in}{0.006944in}}{\pgfqpoint{0.000000in}{0.006944in}}%
\pgfpathcurveto{\pgfqpoint{-0.001842in}{0.006944in}}{\pgfqpoint{-0.003608in}{0.006213in}}{\pgfqpoint{-0.004910in}{0.004910in}}%
\pgfpathcurveto{\pgfqpoint{-0.006213in}{0.003608in}}{\pgfqpoint{-0.006944in}{0.001842in}}{\pgfqpoint{-0.006944in}{0.000000in}}%
\pgfpathcurveto{\pgfqpoint{-0.006944in}{-0.001842in}}{\pgfqpoint{-0.006213in}{-0.003608in}}{\pgfqpoint{-0.004910in}{-0.004910in}}%
\pgfpathcurveto{\pgfqpoint{-0.003608in}{-0.006213in}}{\pgfqpoint{-0.001842in}{-0.006944in}}{\pgfqpoint{0.000000in}{-0.006944in}}%
\pgfpathclose%
\pgfusepath{stroke,fill}%
}%
\begin{pgfscope}%
\pgfsys@transformshift{1.861696in}{1.719090in}%
\pgfsys@useobject{currentmarker}{}%
\end{pgfscope}%
\begin{pgfscope}%
\pgfsys@transformshift{1.852037in}{1.719090in}%
\pgfsys@useobject{currentmarker}{}%
\end{pgfscope}%
\begin{pgfscope}%
\pgfsys@transformshift{1.832719in}{1.719090in}%
\pgfsys@useobject{currentmarker}{}%
\end{pgfscope}%
\begin{pgfscope}%
\pgfsys@transformshift{1.823060in}{1.719090in}%
\pgfsys@useobject{currentmarker}{}%
\end{pgfscope}%
\begin{pgfscope}%
\pgfsys@transformshift{1.813401in}{1.719090in}%
\pgfsys@useobject{currentmarker}{}%
\end{pgfscope}%
\begin{pgfscope}%
\pgfsys@transformshift{1.813401in}{1.719090in}%
\pgfsys@useobject{currentmarker}{}%
\end{pgfscope}%
\begin{pgfscope}%
\pgfsys@transformshift{1.813401in}{1.719090in}%
\pgfsys@useobject{currentmarker}{}%
\end{pgfscope}%
\begin{pgfscope}%
\pgfsys@transformshift{1.813401in}{1.719090in}%
\pgfsys@useobject{currentmarker}{}%
\end{pgfscope}%
\begin{pgfscope}%
\pgfsys@transformshift{1.823060in}{1.719090in}%
\pgfsys@useobject{currentmarker}{}%
\end{pgfscope}%
\begin{pgfscope}%
\pgfsys@transformshift{1.823060in}{1.709503in}%
\pgfsys@useobject{currentmarker}{}%
\end{pgfscope}%
\begin{pgfscope}%
\pgfsys@transformshift{1.823060in}{1.709503in}%
\pgfsys@useobject{currentmarker}{}%
\end{pgfscope}%
\begin{pgfscope}%
\pgfsys@transformshift{1.842378in}{1.709503in}%
\pgfsys@useobject{currentmarker}{}%
\end{pgfscope}%
\begin{pgfscope}%
\pgfsys@transformshift{1.852037in}{1.699917in}%
\pgfsys@useobject{currentmarker}{}%
\end{pgfscope}%
\begin{pgfscope}%
\pgfsys@transformshift{1.852037in}{1.690330in}%
\pgfsys@useobject{currentmarker}{}%
\end{pgfscope}%
\begin{pgfscope}%
\pgfsys@transformshift{1.871355in}{1.690330in}%
\pgfsys@useobject{currentmarker}{}%
\end{pgfscope}%
\begin{pgfscope}%
\pgfsys@transformshift{1.881014in}{1.690330in}%
\pgfsys@useobject{currentmarker}{}%
\end{pgfscope}%
\begin{pgfscope}%
\pgfsys@transformshift{1.919650in}{1.680743in}%
\pgfsys@useobject{currentmarker}{}%
\end{pgfscope}%
\begin{pgfscope}%
\pgfsys@transformshift{1.938968in}{1.671156in}%
\pgfsys@useobject{currentmarker}{}%
\end{pgfscope}%
\begin{pgfscope}%
\pgfsys@transformshift{1.929309in}{1.632809in}%
\pgfsys@useobject{currentmarker}{}%
\end{pgfscope}%
\begin{pgfscope}%
\pgfsys@transformshift{1.919650in}{1.613635in}%
\pgfsys@useobject{currentmarker}{}%
\end{pgfscope}%
\begin{pgfscope}%
\pgfsys@transformshift{1.900332in}{1.575288in}%
\pgfsys@useobject{currentmarker}{}%
\end{pgfscope}%
\begin{pgfscope}%
\pgfsys@transformshift{1.890673in}{1.556114in}%
\pgfsys@useobject{currentmarker}{}%
\end{pgfscope}%
\begin{pgfscope}%
\pgfsys@transformshift{1.881014in}{1.527354in}%
\pgfsys@useobject{currentmarker}{}%
\end{pgfscope}%
\begin{pgfscope}%
\pgfsys@transformshift{1.871355in}{1.508180in}%
\pgfsys@useobject{currentmarker}{}%
\end{pgfscope}%
\begin{pgfscope}%
\pgfsys@transformshift{1.823060in}{1.441072in}%
\pgfsys@useobject{currentmarker}{}%
\end{pgfscope}%
\begin{pgfscope}%
\pgfsys@transformshift{1.861696in}{1.469833in}%
\pgfsys@useobject{currentmarker}{}%
\end{pgfscope}%
\begin{pgfscope}%
\pgfsys@transformshift{1.852037in}{1.441072in}%
\pgfsys@useobject{currentmarker}{}%
\end{pgfscope}%
\begin{pgfscope}%
\pgfsys@transformshift{1.861696in}{1.431485in}%
\pgfsys@useobject{currentmarker}{}%
\end{pgfscope}%
\begin{pgfscope}%
\pgfsys@transformshift{1.861696in}{1.412312in}%
\pgfsys@useobject{currentmarker}{}%
\end{pgfscope}%
\begin{pgfscope}%
\pgfsys@transformshift{1.852037in}{1.402725in}%
\pgfsys@useobject{currentmarker}{}%
\end{pgfscope}%
\begin{pgfscope}%
\pgfsys@transformshift{1.861696in}{1.383551in}%
\pgfsys@useobject{currentmarker}{}%
\end{pgfscope}%
\begin{pgfscope}%
\pgfsys@transformshift{1.871355in}{1.354791in}%
\pgfsys@useobject{currentmarker}{}%
\end{pgfscope}%
\begin{pgfscope}%
\pgfsys@transformshift{1.881014in}{1.345204in}%
\pgfsys@useobject{currentmarker}{}%
\end{pgfscope}%
\begin{pgfscope}%
\pgfsys@transformshift{1.900332in}{1.335617in}%
\pgfsys@useobject{currentmarker}{}%
\end{pgfscope}%
\begin{pgfscope}%
\pgfsys@transformshift{1.919650in}{1.335617in}%
\pgfsys@useobject{currentmarker}{}%
\end{pgfscope}%
\begin{pgfscope}%
\pgfsys@transformshift{1.919650in}{1.326030in}%
\pgfsys@useobject{currentmarker}{}%
\end{pgfscope}%
\begin{pgfscope}%
\pgfsys@transformshift{1.919650in}{1.326030in}%
\pgfsys@useobject{currentmarker}{}%
\end{pgfscope}%
\begin{pgfscope}%
\pgfsys@transformshift{1.938968in}{1.316443in}%
\pgfsys@useobject{currentmarker}{}%
\end{pgfscope}%
\begin{pgfscope}%
\pgfsys@transformshift{1.977604in}{1.287683in}%
\pgfsys@useobject{currentmarker}{}%
\end{pgfscope}%
\begin{pgfscope}%
\pgfsys@transformshift{1.996922in}{1.278096in}%
\pgfsys@useobject{currentmarker}{}%
\end{pgfscope}%
\begin{pgfscope}%
\pgfsys@transformshift{2.025899in}{1.278096in}%
\pgfsys@useobject{currentmarker}{}%
\end{pgfscope}%
\begin{pgfscope}%
\pgfsys@transformshift{2.045217in}{1.258922in}%
\pgfsys@useobject{currentmarker}{}%
\end{pgfscope}%
\begin{pgfscope}%
\pgfsys@transformshift{2.064535in}{1.239749in}%
\pgfsys@useobject{currentmarker}{}%
\end{pgfscope}%
\begin{pgfscope}%
\pgfsys@transformshift{2.103171in}{1.201401in}%
\pgfsys@useobject{currentmarker}{}%
\end{pgfscope}%
\begin{pgfscope}%
\pgfsys@transformshift{2.132148in}{1.143880in}%
\pgfsys@useobject{currentmarker}{}%
\end{pgfscope}%
\begin{pgfscope}%
\pgfsys@transformshift{2.151466in}{1.076773in}%
\pgfsys@useobject{currentmarker}{}%
\end{pgfscope}%
\begin{pgfscope}%
\pgfsys@transformshift{2.151466in}{1.019252in}%
\pgfsys@useobject{currentmarker}{}%
\end{pgfscope}%
\begin{pgfscope}%
\pgfsys@transformshift{2.151466in}{0.952144in}%
\pgfsys@useobject{currentmarker}{}%
\end{pgfscope}%
\begin{pgfscope}%
\pgfsys@transformshift{2.151466in}{0.894623in}%
\pgfsys@useobject{currentmarker}{}%
\end{pgfscope}%
\begin{pgfscope}%
\pgfsys@transformshift{2.141807in}{0.827515in}%
\pgfsys@useobject{currentmarker}{}%
\end{pgfscope}%
\end{pgfscope}%
\begin{pgfscope}%
\pgfpathrectangle{\pgfqpoint{0.847500in}{0.261892in}}{\pgfqpoint{1.642031in}{1.581827in}}%
\pgfusepath{clip}%
\pgfsetbuttcap%
\pgfsetbeveljoin%
\definecolor{currentfill}{rgb}{0.580392,0.403922,0.741176}%
\pgfsetfillcolor{currentfill}%
\pgfsetlinewidth{1.003750pt}%
\definecolor{currentstroke}{rgb}{0.580392,0.403922,0.741176}%
\pgfsetstrokecolor{currentstroke}%
\pgfsetdash{}{0pt}%
\pgfsys@defobject{currentmarker}{\pgfqpoint{-0.006605in}{-0.005618in}}{\pgfqpoint{0.006605in}{0.006944in}}{%
\pgfpathmoveto{\pgfqpoint{0.000000in}{0.006944in}}%
\pgfpathlineto{\pgfqpoint{-0.001559in}{0.002146in}}%
\pgfpathlineto{\pgfqpoint{-0.006605in}{0.002146in}}%
\pgfpathlineto{\pgfqpoint{-0.002523in}{-0.000820in}}%
\pgfpathlineto{\pgfqpoint{-0.004082in}{-0.005618in}}%
\pgfpathlineto{\pgfqpoint{-0.000000in}{-0.002653in}}%
\pgfpathlineto{\pgfqpoint{0.004082in}{-0.005618in}}%
\pgfpathlineto{\pgfqpoint{0.002523in}{-0.000820in}}%
\pgfpathlineto{\pgfqpoint{0.006605in}{0.002146in}}%
\pgfpathlineto{\pgfqpoint{0.001559in}{0.002146in}}%
\pgfpathclose%
\pgfusepath{stroke,fill}%
}%
\begin{pgfscope}%
\pgfsys@transformshift{2.161125in}{0.750820in}%
\pgfsys@useobject{currentmarker}{}%
\end{pgfscope}%
\end{pgfscope}%
\begin{pgfscope}%
\pgfsetrectcap%
\pgfsetmiterjoin%
\pgfsetlinewidth{0.803000pt}%
\definecolor{currentstroke}{rgb}{0.000000,0.000000,0.000000}%
\pgfsetstrokecolor{currentstroke}%
\pgfsetdash{}{0pt}%
\pgfpathmoveto{\pgfqpoint{0.847500in}{0.261892in}}%
\pgfpathlineto{\pgfqpoint{0.847500in}{1.843719in}}%
\pgfusepath{stroke}%
\end{pgfscope}%
\begin{pgfscope}%
\pgfsetrectcap%
\pgfsetmiterjoin%
\pgfsetlinewidth{0.803000pt}%
\definecolor{currentstroke}{rgb}{0.000000,0.000000,0.000000}%
\pgfsetstrokecolor{currentstroke}%
\pgfsetdash{}{0pt}%
\pgfpathmoveto{\pgfqpoint{2.489531in}{0.261892in}}%
\pgfpathlineto{\pgfqpoint{2.489531in}{1.843719in}}%
\pgfusepath{stroke}%
\end{pgfscope}%
\begin{pgfscope}%
\pgfsetrectcap%
\pgfsetmiterjoin%
\pgfsetlinewidth{0.803000pt}%
\definecolor{currentstroke}{rgb}{0.000000,0.000000,0.000000}%
\pgfsetstrokecolor{currentstroke}%
\pgfsetdash{}{0pt}%
\pgfpathmoveto{\pgfqpoint{0.847500in}{0.261892in}}%
\pgfpathlineto{\pgfqpoint{2.489531in}{0.261892in}}%
\pgfusepath{stroke}%
\end{pgfscope}%
\begin{pgfscope}%
\pgfsetrectcap%
\pgfsetmiterjoin%
\pgfsetlinewidth{0.803000pt}%
\definecolor{currentstroke}{rgb}{0.000000,0.000000,0.000000}%
\pgfsetstrokecolor{currentstroke}%
\pgfsetdash{}{0pt}%
\pgfpathmoveto{\pgfqpoint{0.847500in}{1.843719in}}%
\pgfpathlineto{\pgfqpoint{2.489531in}{1.843719in}}%
\pgfusepath{stroke}%
\end{pgfscope}%
\begin{pgfscope}%
\pgfsetbuttcap%
\pgfsetmiterjoin%
\definecolor{currentfill}{rgb}{1.000000,1.000000,1.000000}%
\pgfsetfillcolor{currentfill}%
\pgfsetfillopacity{0.800000}%
\pgfsetlinewidth{1.003750pt}%
\definecolor{currentstroke}{rgb}{0.800000,0.800000,0.800000}%
\pgfsetstrokecolor{currentstroke}%
\pgfsetstrokeopacity{0.800000}%
\pgfsetdash{}{0pt}%
\pgfpathmoveto{\pgfqpoint{0.905833in}{0.922472in}}%
\pgfpathlineto{\pgfqpoint{1.732333in}{0.922472in}}%
\pgfpathquadraticcurveto{\pgfqpoint{1.749000in}{0.922472in}}{\pgfqpoint{1.749000in}{0.939139in}}%
\pgfpathlineto{\pgfqpoint{1.749000in}{1.166472in}}%
\pgfpathquadraticcurveto{\pgfqpoint{1.749000in}{1.183139in}}{\pgfqpoint{1.732333in}{1.183139in}}%
\pgfpathlineto{\pgfqpoint{0.905833in}{1.183139in}}%
\pgfpathquadraticcurveto{\pgfqpoint{0.889167in}{1.183139in}}{\pgfqpoint{0.889167in}{1.166472in}}%
\pgfpathlineto{\pgfqpoint{0.889167in}{0.939139in}}%
\pgfpathquadraticcurveto{\pgfqpoint{0.889167in}{0.922472in}}{\pgfqpoint{0.905833in}{0.922472in}}%
\pgfpathclose%
\pgfusepath{stroke,fill}%
\end{pgfscope}%
\begin{pgfscope}%
\pgfsetrectcap%
\pgfsetroundjoin%
\pgfsetlinewidth{0.401500pt}%
\definecolor{currentstroke}{rgb}{0.121569,0.466667,0.705882}%
\pgfsetstrokecolor{currentstroke}%
\pgfsetdash{}{0pt}%
\pgfpathmoveto{\pgfqpoint{0.922500in}{1.119389in}}%
\pgfpathlineto{\pgfqpoint{1.089167in}{1.119389in}}%
\pgfusepath{stroke}%
\end{pgfscope}%
\begin{pgfscope}%
\pgfsetbuttcap%
\pgfsetroundjoin%
\definecolor{currentfill}{rgb}{0.121569,0.466667,0.705882}%
\pgfsetfillcolor{currentfill}%
\pgfsetlinewidth{1.003750pt}%
\definecolor{currentstroke}{rgb}{0.121569,0.466667,0.705882}%
\pgfsetstrokecolor{currentstroke}%
\pgfsetdash{}{0pt}%
\pgfsys@defobject{currentmarker}{\pgfqpoint{-0.006944in}{-0.006944in}}{\pgfqpoint{0.006944in}{0.006944in}}{%
\pgfpathmoveto{\pgfqpoint{0.000000in}{-0.006944in}}%
\pgfpathcurveto{\pgfqpoint{0.001842in}{-0.006944in}}{\pgfqpoint{0.003608in}{-0.006213in}}{\pgfqpoint{0.004910in}{-0.004910in}}%
\pgfpathcurveto{\pgfqpoint{0.006213in}{-0.003608in}}{\pgfqpoint{0.006944in}{-0.001842in}}{\pgfqpoint{0.006944in}{0.000000in}}%
\pgfpathcurveto{\pgfqpoint{0.006944in}{0.001842in}}{\pgfqpoint{0.006213in}{0.003608in}}{\pgfqpoint{0.004910in}{0.004910in}}%
\pgfpathcurveto{\pgfqpoint{0.003608in}{0.006213in}}{\pgfqpoint{0.001842in}{0.006944in}}{\pgfqpoint{0.000000in}{0.006944in}}%
\pgfpathcurveto{\pgfqpoint{-0.001842in}{0.006944in}}{\pgfqpoint{-0.003608in}{0.006213in}}{\pgfqpoint{-0.004910in}{0.004910in}}%
\pgfpathcurveto{\pgfqpoint{-0.006213in}{0.003608in}}{\pgfqpoint{-0.006944in}{0.001842in}}{\pgfqpoint{-0.006944in}{0.000000in}}%
\pgfpathcurveto{\pgfqpoint{-0.006944in}{-0.001842in}}{\pgfqpoint{-0.006213in}{-0.003608in}}{\pgfqpoint{-0.004910in}{-0.004910in}}%
\pgfpathcurveto{\pgfqpoint{-0.003608in}{-0.006213in}}{\pgfqpoint{-0.001842in}{-0.006944in}}{\pgfqpoint{0.000000in}{-0.006944in}}%
\pgfpathclose%
\pgfusepath{stroke,fill}%
}%
\begin{pgfscope}%
\pgfsys@transformshift{1.005833in}{1.119389in}%
\pgfsys@useobject{currentmarker}{}%
\end{pgfscope}%
\end{pgfscope}%
\begin{pgfscope}%
\definecolor{textcolor}{rgb}{0.000000,0.000000,0.000000}%
\pgfsetstrokecolor{textcolor}%
\pgfsetfillcolor{textcolor}%
\pgftext[x=1.155833in,y=1.090222in,left,base]{\color{textcolor}\sffamily\fontsize{6.000000}{7.200000}\selectfont Main Sequence}%
\end{pgfscope}%
\begin{pgfscope}%
\pgfsetbuttcap%
\pgfsetbeveljoin%
\definecolor{currentfill}{rgb}{0.580392,0.403922,0.741176}%
\pgfsetfillcolor{currentfill}%
\pgfsetlinewidth{1.003750pt}%
\definecolor{currentstroke}{rgb}{0.580392,0.403922,0.741176}%
\pgfsetstrokecolor{currentstroke}%
\pgfsetdash{}{0pt}%
\pgfsys@defobject{currentmarker}{\pgfqpoint{-0.006605in}{-0.005618in}}{\pgfqpoint{0.006605in}{0.006944in}}{%
\pgfpathmoveto{\pgfqpoint{0.000000in}{0.006944in}}%
\pgfpathlineto{\pgfqpoint{-0.001559in}{0.002146in}}%
\pgfpathlineto{\pgfqpoint{-0.006605in}{0.002146in}}%
\pgfpathlineto{\pgfqpoint{-0.002523in}{-0.000820in}}%
\pgfpathlineto{\pgfqpoint{-0.004082in}{-0.005618in}}%
\pgfpathlineto{\pgfqpoint{-0.000000in}{-0.002653in}}%
\pgfpathlineto{\pgfqpoint{0.004082in}{-0.005618in}}%
\pgfpathlineto{\pgfqpoint{0.002523in}{-0.000820in}}%
\pgfpathlineto{\pgfqpoint{0.006605in}{0.002146in}}%
\pgfpathlineto{\pgfqpoint{0.001559in}{0.002146in}}%
\pgfpathclose%
\pgfusepath{stroke,fill}%
}%
\begin{pgfscope}%
\pgfsys@transformshift{1.005833in}{1.001972in}%
\pgfsys@useobject{currentmarker}{}%
\end{pgfscope}%
\end{pgfscope}%
\begin{pgfscope}%
\definecolor{textcolor}{rgb}{0.000000,0.000000,0.000000}%
\pgfsetstrokecolor{textcolor}%
\pgfsetfillcolor{textcolor}%
\pgftext[x=1.155833in,y=0.972805in,left,base]{\color{textcolor}\sffamily\fontsize{6.000000}{7.200000}\selectfont NIR Object}%
\end{pgfscope}%
\begin{pgfscope}%
\pgfsetbuttcap%
\pgfsetmiterjoin%
\definecolor{currentfill}{rgb}{1.000000,1.000000,1.000000}%
\pgfsetfillcolor{currentfill}%
\pgfsetlinewidth{0.000000pt}%
\definecolor{currentstroke}{rgb}{0.000000,0.000000,0.000000}%
\pgfsetstrokecolor{currentstroke}%
\pgfsetstrokeopacity{0.000000}%
\pgfsetdash{}{0pt}%
\pgfpathmoveto{\pgfqpoint{2.653734in}{0.261892in}}%
\pgfpathlineto{\pgfqpoint{4.295766in}{0.261892in}}%
\pgfpathlineto{\pgfqpoint{4.295766in}{1.843719in}}%
\pgfpathlineto{\pgfqpoint{2.653734in}{1.843719in}}%
\pgfpathclose%
\pgfusepath{fill}%
\end{pgfscope}%
\begin{pgfscope}%
\pgfpathrectangle{\pgfqpoint{2.653734in}{0.261892in}}{\pgfqpoint{1.642031in}{1.581827in}}%
\pgfusepath{clip}%
\pgfsetrectcap%
\pgfsetroundjoin%
\pgfsetlinewidth{0.281050pt}%
\definecolor{currentstroke}{rgb}{0.690196,0.690196,0.690196}%
\pgfsetstrokecolor{currentstroke}%
\pgfsetdash{}{0pt}%
\pgfpathmoveto{\pgfqpoint{2.750324in}{0.261892in}}%
\pgfpathlineto{\pgfqpoint{2.750324in}{1.843719in}}%
\pgfusepath{stroke}%
\end{pgfscope}%
\begin{pgfscope}%
\pgfsetbuttcap%
\pgfsetroundjoin%
\definecolor{currentfill}{rgb}{0.000000,0.000000,0.000000}%
\pgfsetfillcolor{currentfill}%
\pgfsetlinewidth{0.803000pt}%
\definecolor{currentstroke}{rgb}{0.000000,0.000000,0.000000}%
\pgfsetstrokecolor{currentstroke}%
\pgfsetdash{}{0pt}%
\pgfsys@defobject{currentmarker}{\pgfqpoint{0.000000in}{-0.048611in}}{\pgfqpoint{0.000000in}{0.000000in}}{%
\pgfpathmoveto{\pgfqpoint{0.000000in}{0.000000in}}%
\pgfpathlineto{\pgfqpoint{0.000000in}{-0.048611in}}%
\pgfusepath{stroke,fill}%
}%
\begin{pgfscope}%
\pgfsys@transformshift{2.750324in}{0.261892in}%
\pgfsys@useobject{currentmarker}{}%
\end{pgfscope}%
\end{pgfscope}%
\begin{pgfscope}%
\definecolor{textcolor}{rgb}{0.000000,0.000000,0.000000}%
\pgfsetstrokecolor{textcolor}%
\pgfsetfillcolor{textcolor}%
\pgftext[x=2.750324in,y=0.164670in,,top]{\color{textcolor}\sffamily\fontsize{8.000000}{9.600000}\selectfont −1.0}%
\end{pgfscope}%
\begin{pgfscope}%
\pgfpathrectangle{\pgfqpoint{2.653734in}{0.261892in}}{\pgfqpoint{1.642031in}{1.581827in}}%
\pgfusepath{clip}%
\pgfsetrectcap%
\pgfsetroundjoin%
\pgfsetlinewidth{0.281050pt}%
\definecolor{currentstroke}{rgb}{0.690196,0.690196,0.690196}%
\pgfsetstrokecolor{currentstroke}%
\pgfsetdash{}{0pt}%
\pgfpathmoveto{\pgfqpoint{3.233275in}{0.261892in}}%
\pgfpathlineto{\pgfqpoint{3.233275in}{1.843719in}}%
\pgfusepath{stroke}%
\end{pgfscope}%
\begin{pgfscope}%
\pgfsetbuttcap%
\pgfsetroundjoin%
\definecolor{currentfill}{rgb}{0.000000,0.000000,0.000000}%
\pgfsetfillcolor{currentfill}%
\pgfsetlinewidth{0.803000pt}%
\definecolor{currentstroke}{rgb}{0.000000,0.000000,0.000000}%
\pgfsetstrokecolor{currentstroke}%
\pgfsetdash{}{0pt}%
\pgfsys@defobject{currentmarker}{\pgfqpoint{0.000000in}{-0.048611in}}{\pgfqpoint{0.000000in}{0.000000in}}{%
\pgfpathmoveto{\pgfqpoint{0.000000in}{0.000000in}}%
\pgfpathlineto{\pgfqpoint{0.000000in}{-0.048611in}}%
\pgfusepath{stroke,fill}%
}%
\begin{pgfscope}%
\pgfsys@transformshift{3.233275in}{0.261892in}%
\pgfsys@useobject{currentmarker}{}%
\end{pgfscope}%
\end{pgfscope}%
\begin{pgfscope}%
\definecolor{textcolor}{rgb}{0.000000,0.000000,0.000000}%
\pgfsetstrokecolor{textcolor}%
\pgfsetfillcolor{textcolor}%
\pgftext[x=3.233275in,y=0.164670in,,top]{\color{textcolor}\sffamily\fontsize{8.000000}{9.600000}\selectfont −0.5}%
\end{pgfscope}%
\begin{pgfscope}%
\pgfpathrectangle{\pgfqpoint{2.653734in}{0.261892in}}{\pgfqpoint{1.642031in}{1.581827in}}%
\pgfusepath{clip}%
\pgfsetrectcap%
\pgfsetroundjoin%
\pgfsetlinewidth{0.281050pt}%
\definecolor{currentstroke}{rgb}{0.690196,0.690196,0.690196}%
\pgfsetstrokecolor{currentstroke}%
\pgfsetdash{}{0pt}%
\pgfpathmoveto{\pgfqpoint{3.716225in}{0.261892in}}%
\pgfpathlineto{\pgfqpoint{3.716225in}{1.843719in}}%
\pgfusepath{stroke}%
\end{pgfscope}%
\begin{pgfscope}%
\pgfsetbuttcap%
\pgfsetroundjoin%
\definecolor{currentfill}{rgb}{0.000000,0.000000,0.000000}%
\pgfsetfillcolor{currentfill}%
\pgfsetlinewidth{0.803000pt}%
\definecolor{currentstroke}{rgb}{0.000000,0.000000,0.000000}%
\pgfsetstrokecolor{currentstroke}%
\pgfsetdash{}{0pt}%
\pgfsys@defobject{currentmarker}{\pgfqpoint{0.000000in}{-0.048611in}}{\pgfqpoint{0.000000in}{0.000000in}}{%
\pgfpathmoveto{\pgfqpoint{0.000000in}{0.000000in}}%
\pgfpathlineto{\pgfqpoint{0.000000in}{-0.048611in}}%
\pgfusepath{stroke,fill}%
}%
\begin{pgfscope}%
\pgfsys@transformshift{3.716225in}{0.261892in}%
\pgfsys@useobject{currentmarker}{}%
\end{pgfscope}%
\end{pgfscope}%
\begin{pgfscope}%
\definecolor{textcolor}{rgb}{0.000000,0.000000,0.000000}%
\pgfsetstrokecolor{textcolor}%
\pgfsetfillcolor{textcolor}%
\pgftext[x=3.716225in,y=0.164670in,,top]{\color{textcolor}\sffamily\fontsize{8.000000}{9.600000}\selectfont 0.0}%
\end{pgfscope}%
\begin{pgfscope}%
\pgfpathrectangle{\pgfqpoint{2.653734in}{0.261892in}}{\pgfqpoint{1.642031in}{1.581827in}}%
\pgfusepath{clip}%
\pgfsetrectcap%
\pgfsetroundjoin%
\pgfsetlinewidth{0.281050pt}%
\definecolor{currentstroke}{rgb}{0.690196,0.690196,0.690196}%
\pgfsetstrokecolor{currentstroke}%
\pgfsetdash{}{0pt}%
\pgfpathmoveto{\pgfqpoint{4.199176in}{0.261892in}}%
\pgfpathlineto{\pgfqpoint{4.199176in}{1.843719in}}%
\pgfusepath{stroke}%
\end{pgfscope}%
\begin{pgfscope}%
\pgfsetbuttcap%
\pgfsetroundjoin%
\definecolor{currentfill}{rgb}{0.000000,0.000000,0.000000}%
\pgfsetfillcolor{currentfill}%
\pgfsetlinewidth{0.803000pt}%
\definecolor{currentstroke}{rgb}{0.000000,0.000000,0.000000}%
\pgfsetstrokecolor{currentstroke}%
\pgfsetdash{}{0pt}%
\pgfsys@defobject{currentmarker}{\pgfqpoint{0.000000in}{-0.048611in}}{\pgfqpoint{0.000000in}{0.000000in}}{%
\pgfpathmoveto{\pgfqpoint{0.000000in}{0.000000in}}%
\pgfpathlineto{\pgfqpoint{0.000000in}{-0.048611in}}%
\pgfusepath{stroke,fill}%
}%
\begin{pgfscope}%
\pgfsys@transformshift{4.199176in}{0.261892in}%
\pgfsys@useobject{currentmarker}{}%
\end{pgfscope}%
\end{pgfscope}%
\begin{pgfscope}%
\definecolor{textcolor}{rgb}{0.000000,0.000000,0.000000}%
\pgfsetstrokecolor{textcolor}%
\pgfsetfillcolor{textcolor}%
\pgftext[x=4.199176in,y=0.164670in,,top]{\color{textcolor}\sffamily\fontsize{8.000000}{9.600000}\selectfont 0.5}%
\end{pgfscope}%
\begin{pgfscope}%
\definecolor{textcolor}{rgb}{0.000000,0.000000,0.000000}%
\pgfsetstrokecolor{textcolor}%
\pgfsetfillcolor{textcolor}%
\pgftext[x=3.474750in,y=0.010448in,,top]{\color{textcolor}\sffamily\fontsize{8.000000}{9.600000}\selectfont \(\displaystyle (H-K)_0\)}%
\end{pgfscope}%
\begin{pgfscope}%
\pgfpathrectangle{\pgfqpoint{2.653734in}{0.261892in}}{\pgfqpoint{1.642031in}{1.581827in}}%
\pgfusepath{clip}%
\pgfsetbuttcap%
\pgfsetmiterjoin%
\definecolor{currentfill}{rgb}{0.000000,0.000000,0.000000}%
\pgfsetfillcolor{currentfill}%
\pgfsetlinewidth{1.003750pt}%
\definecolor{currentstroke}{rgb}{0.000000,0.000000,0.000000}%
\pgfsetstrokecolor{currentstroke}%
\pgfsetdash{}{0pt}%
\pgfpathmoveto{\pgfqpoint{4.279699in}{0.795857in}}%
\pgfpathlineto{\pgfqpoint{4.252178in}{0.866554in}}%
\pgfpathlineto{\pgfqpoint{4.233934in}{0.851596in}}%
\pgfpathlineto{\pgfqpoint{4.006368in}{1.125012in}}%
\pgfpathlineto{\pgfqpoint{4.005623in}{1.124401in}}%
\pgfpathlineto{\pgfqpoint{4.233189in}{0.850985in}}%
\pgfpathlineto{\pgfqpoint{4.214945in}{0.836026in}}%
\pgfpathclose%
\pgfusepath{stroke,fill}%
\end{pgfscope}%
\begin{pgfscope}%
\pgfpathrectangle{\pgfqpoint{2.653734in}{0.261892in}}{\pgfqpoint{1.642031in}{1.581827in}}%
\pgfusepath{clip}%
\pgfsetbuttcap%
\pgfsetmiterjoin%
\definecolor{currentfill}{rgb}{0.000000,0.000000,0.000000}%
\pgfsetfillcolor{currentfill}%
\pgfsetlinewidth{1.003750pt}%
\definecolor{currentstroke}{rgb}{0.000000,0.000000,0.000000}%
\pgfsetstrokecolor{currentstroke}%
\pgfsetdash{}{0pt}%
\pgfpathmoveto{\pgfqpoint{3.875496in}{0.861192in}}%
\pgfpathlineto{\pgfqpoint{3.886504in}{0.832913in}}%
\pgfpathlineto{\pgfqpoint{3.893579in}{0.838714in}}%
\pgfpathlineto{\pgfqpoint{3.966987in}{0.750515in}}%
\pgfpathlineto{\pgfqpoint{3.967732in}{0.751126in}}%
\pgfpathlineto{\pgfqpoint{3.894323in}{0.839324in}}%
\pgfpathlineto{\pgfqpoint{3.901398in}{0.845125in}}%
\pgfpathclose%
\pgfusepath{stroke,fill}%
\end{pgfscope}%
\begin{pgfscope}%
\pgfpathrectangle{\pgfqpoint{2.653734in}{0.261892in}}{\pgfqpoint{1.642031in}{1.581827in}}%
\pgfusepath{clip}%
\pgfsetrectcap%
\pgfsetroundjoin%
\pgfsetlinewidth{0.401500pt}%
\definecolor{currentstroke}{rgb}{1.000000,0.498039,0.054902}%
\pgfsetstrokecolor{currentstroke}%
\pgfsetdash{}{0pt}%
\pgfpathmoveto{\pgfqpoint{3.754861in}{1.134294in}}%
\pgfpathlineto{\pgfqpoint{3.764520in}{1.134294in}}%
\pgfpathlineto{\pgfqpoint{3.783838in}{1.134294in}}%
\pgfpathlineto{\pgfqpoint{3.803156in}{1.124707in}}%
\pgfpathlineto{\pgfqpoint{3.870769in}{1.124707in}}%
\pgfpathlineto{\pgfqpoint{3.899746in}{1.134294in}}%
\pgfpathlineto{\pgfqpoint{3.919064in}{1.134294in}}%
\pgfpathlineto{\pgfqpoint{3.928723in}{1.134294in}}%
\pgfpathlineto{\pgfqpoint{3.948041in}{1.134294in}}%
\pgfpathlineto{\pgfqpoint{3.957700in}{1.124707in}}%
\pgfpathlineto{\pgfqpoint{3.986677in}{1.105533in}}%
\pgfpathlineto{\pgfqpoint{3.996336in}{1.086359in}}%
\pgfpathlineto{\pgfqpoint{3.996336in}{1.067186in}}%
\pgfpathlineto{\pgfqpoint{3.841792in}{0.894623in}}%
\pgfpathlineto{\pgfqpoint{3.986677in}{1.019252in}}%
\pgfpathlineto{\pgfqpoint{3.977018in}{1.009665in}}%
\pgfpathlineto{\pgfqpoint{3.938382in}{0.952144in}}%
\pgfpathlineto{\pgfqpoint{3.928723in}{0.942557in}}%
\pgfpathlineto{\pgfqpoint{3.899746in}{0.913796in}}%
\pgfpathlineto{\pgfqpoint{3.841792in}{0.865862in}}%
\pgfpathlineto{\pgfqpoint{3.764520in}{0.808341in}}%
\pgfpathlineto{\pgfqpoint{3.648612in}{0.731647in}}%
\pgfpathlineto{\pgfqpoint{3.561681in}{0.674126in}}%
\pgfpathlineto{\pgfqpoint{3.474750in}{0.616605in}}%
\pgfpathlineto{\pgfqpoint{3.271911in}{0.491976in}}%
\pgfusepath{stroke}%
\end{pgfscope}%
\begin{pgfscope}%
\pgfpathrectangle{\pgfqpoint{2.653734in}{0.261892in}}{\pgfqpoint{1.642031in}{1.581827in}}%
\pgfusepath{clip}%
\pgfsetbuttcap%
\pgfsetroundjoin%
\definecolor{currentfill}{rgb}{1.000000,0.498039,0.054902}%
\pgfsetfillcolor{currentfill}%
\pgfsetlinewidth{1.003750pt}%
\definecolor{currentstroke}{rgb}{1.000000,0.498039,0.054902}%
\pgfsetstrokecolor{currentstroke}%
\pgfsetdash{}{0pt}%
\pgfsys@defobject{currentmarker}{\pgfqpoint{-0.006944in}{-0.006944in}}{\pgfqpoint{0.006944in}{0.006944in}}{%
\pgfpathmoveto{\pgfqpoint{0.000000in}{-0.006944in}}%
\pgfpathcurveto{\pgfqpoint{0.001842in}{-0.006944in}}{\pgfqpoint{0.003608in}{-0.006213in}}{\pgfqpoint{0.004910in}{-0.004910in}}%
\pgfpathcurveto{\pgfqpoint{0.006213in}{-0.003608in}}{\pgfqpoint{0.006944in}{-0.001842in}}{\pgfqpoint{0.006944in}{0.000000in}}%
\pgfpathcurveto{\pgfqpoint{0.006944in}{0.001842in}}{\pgfqpoint{0.006213in}{0.003608in}}{\pgfqpoint{0.004910in}{0.004910in}}%
\pgfpathcurveto{\pgfqpoint{0.003608in}{0.006213in}}{\pgfqpoint{0.001842in}{0.006944in}}{\pgfqpoint{0.000000in}{0.006944in}}%
\pgfpathcurveto{\pgfqpoint{-0.001842in}{0.006944in}}{\pgfqpoint{-0.003608in}{0.006213in}}{\pgfqpoint{-0.004910in}{0.004910in}}%
\pgfpathcurveto{\pgfqpoint{-0.006213in}{0.003608in}}{\pgfqpoint{-0.006944in}{0.001842in}}{\pgfqpoint{-0.006944in}{0.000000in}}%
\pgfpathcurveto{\pgfqpoint{-0.006944in}{-0.001842in}}{\pgfqpoint{-0.006213in}{-0.003608in}}{\pgfqpoint{-0.004910in}{-0.004910in}}%
\pgfpathcurveto{\pgfqpoint{-0.003608in}{-0.006213in}}{\pgfqpoint{-0.001842in}{-0.006944in}}{\pgfqpoint{0.000000in}{-0.006944in}}%
\pgfpathclose%
\pgfusepath{stroke,fill}%
}%
\begin{pgfscope}%
\pgfsys@transformshift{3.754861in}{1.134294in}%
\pgfsys@useobject{currentmarker}{}%
\end{pgfscope}%
\begin{pgfscope}%
\pgfsys@transformshift{3.764520in}{1.134294in}%
\pgfsys@useobject{currentmarker}{}%
\end{pgfscope}%
\begin{pgfscope}%
\pgfsys@transformshift{3.783838in}{1.134294in}%
\pgfsys@useobject{currentmarker}{}%
\end{pgfscope}%
\begin{pgfscope}%
\pgfsys@transformshift{3.803156in}{1.124707in}%
\pgfsys@useobject{currentmarker}{}%
\end{pgfscope}%
\begin{pgfscope}%
\pgfsys@transformshift{3.870769in}{1.124707in}%
\pgfsys@useobject{currentmarker}{}%
\end{pgfscope}%
\begin{pgfscope}%
\pgfsys@transformshift{3.899746in}{1.134294in}%
\pgfsys@useobject{currentmarker}{}%
\end{pgfscope}%
\begin{pgfscope}%
\pgfsys@transformshift{3.919064in}{1.134294in}%
\pgfsys@useobject{currentmarker}{}%
\end{pgfscope}%
\begin{pgfscope}%
\pgfsys@transformshift{3.928723in}{1.134294in}%
\pgfsys@useobject{currentmarker}{}%
\end{pgfscope}%
\begin{pgfscope}%
\pgfsys@transformshift{3.948041in}{1.134294in}%
\pgfsys@useobject{currentmarker}{}%
\end{pgfscope}%
\begin{pgfscope}%
\pgfsys@transformshift{3.957700in}{1.124707in}%
\pgfsys@useobject{currentmarker}{}%
\end{pgfscope}%
\begin{pgfscope}%
\pgfsys@transformshift{3.986677in}{1.105533in}%
\pgfsys@useobject{currentmarker}{}%
\end{pgfscope}%
\begin{pgfscope}%
\pgfsys@transformshift{3.996336in}{1.086359in}%
\pgfsys@useobject{currentmarker}{}%
\end{pgfscope}%
\begin{pgfscope}%
\pgfsys@transformshift{3.996336in}{1.067186in}%
\pgfsys@useobject{currentmarker}{}%
\end{pgfscope}%
\begin{pgfscope}%
\pgfsys@transformshift{3.841792in}{0.894623in}%
\pgfsys@useobject{currentmarker}{}%
\end{pgfscope}%
\begin{pgfscope}%
\pgfsys@transformshift{3.986677in}{1.019252in}%
\pgfsys@useobject{currentmarker}{}%
\end{pgfscope}%
\begin{pgfscope}%
\pgfsys@transformshift{3.977018in}{1.009665in}%
\pgfsys@useobject{currentmarker}{}%
\end{pgfscope}%
\begin{pgfscope}%
\pgfsys@transformshift{3.938382in}{0.952144in}%
\pgfsys@useobject{currentmarker}{}%
\end{pgfscope}%
\begin{pgfscope}%
\pgfsys@transformshift{3.928723in}{0.942557in}%
\pgfsys@useobject{currentmarker}{}%
\end{pgfscope}%
\begin{pgfscope}%
\pgfsys@transformshift{3.899746in}{0.913796in}%
\pgfsys@useobject{currentmarker}{}%
\end{pgfscope}%
\begin{pgfscope}%
\pgfsys@transformshift{3.841792in}{0.865862in}%
\pgfsys@useobject{currentmarker}{}%
\end{pgfscope}%
\begin{pgfscope}%
\pgfsys@transformshift{3.764520in}{0.808341in}%
\pgfsys@useobject{currentmarker}{}%
\end{pgfscope}%
\begin{pgfscope}%
\pgfsys@transformshift{3.648612in}{0.731647in}%
\pgfsys@useobject{currentmarker}{}%
\end{pgfscope}%
\begin{pgfscope}%
\pgfsys@transformshift{3.561681in}{0.674126in}%
\pgfsys@useobject{currentmarker}{}%
\end{pgfscope}%
\begin{pgfscope}%
\pgfsys@transformshift{3.474750in}{0.616605in}%
\pgfsys@useobject{currentmarker}{}%
\end{pgfscope}%
\begin{pgfscope}%
\pgfsys@transformshift{3.271911in}{0.491976in}%
\pgfsys@useobject{currentmarker}{}%
\end{pgfscope}%
\end{pgfscope}%
\begin{pgfscope}%
\pgfpathrectangle{\pgfqpoint{2.653734in}{0.261892in}}{\pgfqpoint{1.642031in}{1.581827in}}%
\pgfusepath{clip}%
\pgfsetbuttcap%
\pgfsetbeveljoin%
\definecolor{currentfill}{rgb}{0.580392,0.403922,0.741176}%
\pgfsetfillcolor{currentfill}%
\pgfsetlinewidth{1.003750pt}%
\definecolor{currentstroke}{rgb}{0.580392,0.403922,0.741176}%
\pgfsetstrokecolor{currentstroke}%
\pgfsetdash{}{0pt}%
\pgfsys@defobject{currentmarker}{\pgfqpoint{-0.006605in}{-0.005618in}}{\pgfqpoint{0.006605in}{0.006944in}}{%
\pgfpathmoveto{\pgfqpoint{0.000000in}{0.006944in}}%
\pgfpathlineto{\pgfqpoint{-0.001559in}{0.002146in}}%
\pgfpathlineto{\pgfqpoint{-0.006605in}{0.002146in}}%
\pgfpathlineto{\pgfqpoint{-0.002523in}{-0.000820in}}%
\pgfpathlineto{\pgfqpoint{-0.004082in}{-0.005618in}}%
\pgfpathlineto{\pgfqpoint{-0.000000in}{-0.002653in}}%
\pgfpathlineto{\pgfqpoint{0.004082in}{-0.005618in}}%
\pgfpathlineto{\pgfqpoint{0.002523in}{-0.000820in}}%
\pgfpathlineto{\pgfqpoint{0.006605in}{0.002146in}}%
\pgfpathlineto{\pgfqpoint{0.001559in}{0.002146in}}%
\pgfpathclose%
\pgfusepath{stroke,fill}%
}%
\begin{pgfscope}%
\pgfsys@transformshift{3.967359in}{0.750820in}%
\pgfsys@useobject{currentmarker}{}%
\end{pgfscope}%
\end{pgfscope}%
\begin{pgfscope}%
\pgfsetrectcap%
\pgfsetmiterjoin%
\pgfsetlinewidth{0.803000pt}%
\definecolor{currentstroke}{rgb}{0.000000,0.000000,0.000000}%
\pgfsetstrokecolor{currentstroke}%
\pgfsetdash{}{0pt}%
\pgfpathmoveto{\pgfqpoint{2.653734in}{0.261892in}}%
\pgfpathlineto{\pgfqpoint{2.653734in}{1.843719in}}%
\pgfusepath{stroke}%
\end{pgfscope}%
\begin{pgfscope}%
\pgfsetrectcap%
\pgfsetmiterjoin%
\pgfsetlinewidth{0.803000pt}%
\definecolor{currentstroke}{rgb}{0.000000,0.000000,0.000000}%
\pgfsetstrokecolor{currentstroke}%
\pgfsetdash{}{0pt}%
\pgfpathmoveto{\pgfqpoint{4.295766in}{0.261892in}}%
\pgfpathlineto{\pgfqpoint{4.295766in}{1.843719in}}%
\pgfusepath{stroke}%
\end{pgfscope}%
\begin{pgfscope}%
\pgfsetrectcap%
\pgfsetmiterjoin%
\pgfsetlinewidth{0.803000pt}%
\definecolor{currentstroke}{rgb}{0.000000,0.000000,0.000000}%
\pgfsetstrokecolor{currentstroke}%
\pgfsetdash{}{0pt}%
\pgfpathmoveto{\pgfqpoint{2.653734in}{0.261892in}}%
\pgfpathlineto{\pgfqpoint{4.295766in}{0.261892in}}%
\pgfusepath{stroke}%
\end{pgfscope}%
\begin{pgfscope}%
\pgfsetrectcap%
\pgfsetmiterjoin%
\pgfsetlinewidth{0.803000pt}%
\definecolor{currentstroke}{rgb}{0.000000,0.000000,0.000000}%
\pgfsetstrokecolor{currentstroke}%
\pgfsetdash{}{0pt}%
\pgfpathmoveto{\pgfqpoint{2.653734in}{1.843719in}}%
\pgfpathlineto{\pgfqpoint{4.295766in}{1.843719in}}%
\pgfusepath{stroke}%
\end{pgfscope}%
\begin{pgfscope}%
\pgfsetbuttcap%
\pgfsetmiterjoin%
\definecolor{currentfill}{rgb}{1.000000,1.000000,1.000000}%
\pgfsetfillcolor{currentfill}%
\pgfsetfillopacity{0.800000}%
\pgfsetlinewidth{1.003750pt}%
\definecolor{currentstroke}{rgb}{0.800000,0.800000,0.800000}%
\pgfsetstrokecolor{currentstroke}%
\pgfsetstrokeopacity{0.800000}%
\pgfsetdash{}{0pt}%
\pgfpathmoveto{\pgfqpoint{2.712068in}{0.923097in}}%
\pgfpathlineto{\pgfqpoint{3.398068in}{0.923097in}}%
\pgfpathquadraticcurveto{\pgfqpoint{3.414734in}{0.923097in}}{\pgfqpoint{3.414734in}{0.939764in}}%
\pgfpathlineto{\pgfqpoint{3.414734in}{1.165847in}}%
\pgfpathquadraticcurveto{\pgfqpoint{3.414734in}{1.182514in}}{\pgfqpoint{3.398068in}{1.182514in}}%
\pgfpathlineto{\pgfqpoint{2.712068in}{1.182514in}}%
\pgfpathquadraticcurveto{\pgfqpoint{2.695401in}{1.182514in}}{\pgfqpoint{2.695401in}{1.165847in}}%
\pgfpathlineto{\pgfqpoint{2.695401in}{0.939764in}}%
\pgfpathquadraticcurveto{\pgfqpoint{2.695401in}{0.923097in}}{\pgfqpoint{2.712068in}{0.923097in}}%
\pgfpathclose%
\pgfusepath{stroke,fill}%
\end{pgfscope}%
\begin{pgfscope}%
\pgfsetrectcap%
\pgfsetroundjoin%
\pgfsetlinewidth{0.401500pt}%
\definecolor{currentstroke}{rgb}{1.000000,0.498039,0.054902}%
\pgfsetstrokecolor{currentstroke}%
\pgfsetdash{}{0pt}%
\pgfpathmoveto{\pgfqpoint{2.728734in}{1.120014in}}%
\pgfpathlineto{\pgfqpoint{2.895401in}{1.120014in}}%
\pgfusepath{stroke}%
\end{pgfscope}%
\begin{pgfscope}%
\pgfsetbuttcap%
\pgfsetroundjoin%
\definecolor{currentfill}{rgb}{1.000000,0.498039,0.054902}%
\pgfsetfillcolor{currentfill}%
\pgfsetlinewidth{1.003750pt}%
\definecolor{currentstroke}{rgb}{1.000000,0.498039,0.054902}%
\pgfsetstrokecolor{currentstroke}%
\pgfsetdash{}{0pt}%
\pgfsys@defobject{currentmarker}{\pgfqpoint{-0.006944in}{-0.006944in}}{\pgfqpoint{0.006944in}{0.006944in}}{%
\pgfpathmoveto{\pgfqpoint{0.000000in}{-0.006944in}}%
\pgfpathcurveto{\pgfqpoint{0.001842in}{-0.006944in}}{\pgfqpoint{0.003608in}{-0.006213in}}{\pgfqpoint{0.004910in}{-0.004910in}}%
\pgfpathcurveto{\pgfqpoint{0.006213in}{-0.003608in}}{\pgfqpoint{0.006944in}{-0.001842in}}{\pgfqpoint{0.006944in}{0.000000in}}%
\pgfpathcurveto{\pgfqpoint{0.006944in}{0.001842in}}{\pgfqpoint{0.006213in}{0.003608in}}{\pgfqpoint{0.004910in}{0.004910in}}%
\pgfpathcurveto{\pgfqpoint{0.003608in}{0.006213in}}{\pgfqpoint{0.001842in}{0.006944in}}{\pgfqpoint{0.000000in}{0.006944in}}%
\pgfpathcurveto{\pgfqpoint{-0.001842in}{0.006944in}}{\pgfqpoint{-0.003608in}{0.006213in}}{\pgfqpoint{-0.004910in}{0.004910in}}%
\pgfpathcurveto{\pgfqpoint{-0.006213in}{0.003608in}}{\pgfqpoint{-0.006944in}{0.001842in}}{\pgfqpoint{-0.006944in}{0.000000in}}%
\pgfpathcurveto{\pgfqpoint{-0.006944in}{-0.001842in}}{\pgfqpoint{-0.006213in}{-0.003608in}}{\pgfqpoint{-0.004910in}{-0.004910in}}%
\pgfpathcurveto{\pgfqpoint{-0.003608in}{-0.006213in}}{\pgfqpoint{-0.001842in}{-0.006944in}}{\pgfqpoint{0.000000in}{-0.006944in}}%
\pgfpathclose%
\pgfusepath{stroke,fill}%
}%
\begin{pgfscope}%
\pgfsys@transformshift{2.812068in}{1.120014in}%
\pgfsys@useobject{currentmarker}{}%
\end{pgfscope}%
\end{pgfscope}%
\begin{pgfscope}%
\definecolor{textcolor}{rgb}{0.000000,0.000000,0.000000}%
\pgfsetstrokecolor{textcolor}%
\pgfsetfillcolor{textcolor}%
\pgftext[x=2.962068in,y=1.090847in,left,base]{\color{textcolor}\sffamily\fontsize{6.000000}{7.200000}\selectfont Giants}%
\end{pgfscope}%
\begin{pgfscope}%
\pgfsetbuttcap%
\pgfsetbeveljoin%
\definecolor{currentfill}{rgb}{0.580392,0.403922,0.741176}%
\pgfsetfillcolor{currentfill}%
\pgfsetlinewidth{1.003750pt}%
\definecolor{currentstroke}{rgb}{0.580392,0.403922,0.741176}%
\pgfsetstrokecolor{currentstroke}%
\pgfsetdash{}{0pt}%
\pgfsys@defobject{currentmarker}{\pgfqpoint{-0.006605in}{-0.005618in}}{\pgfqpoint{0.006605in}{0.006944in}}{%
\pgfpathmoveto{\pgfqpoint{0.000000in}{0.006944in}}%
\pgfpathlineto{\pgfqpoint{-0.001559in}{0.002146in}}%
\pgfpathlineto{\pgfqpoint{-0.006605in}{0.002146in}}%
\pgfpathlineto{\pgfqpoint{-0.002523in}{-0.000820in}}%
\pgfpathlineto{\pgfqpoint{-0.004082in}{-0.005618in}}%
\pgfpathlineto{\pgfqpoint{-0.000000in}{-0.002653in}}%
\pgfpathlineto{\pgfqpoint{0.004082in}{-0.005618in}}%
\pgfpathlineto{\pgfqpoint{0.002523in}{-0.000820in}}%
\pgfpathlineto{\pgfqpoint{0.006605in}{0.002146in}}%
\pgfpathlineto{\pgfqpoint{0.001559in}{0.002146in}}%
\pgfpathclose%
\pgfusepath{stroke,fill}%
}%
\begin{pgfscope}%
\pgfsys@transformshift{2.812068in}{1.002597in}%
\pgfsys@useobject{currentmarker}{}%
\end{pgfscope}%
\end{pgfscope}%
\begin{pgfscope}%
\definecolor{textcolor}{rgb}{0.000000,0.000000,0.000000}%
\pgfsetstrokecolor{textcolor}%
\pgfsetfillcolor{textcolor}%
\pgftext[x=2.962068in,y=0.973430in,left,base]{\color{textcolor}\sffamily\fontsize{6.000000}{7.200000}\selectfont NIR Object}%
\end{pgfscope}%
\begin{pgfscope}%
\pgfsetbuttcap%
\pgfsetmiterjoin%
\definecolor{currentfill}{rgb}{1.000000,1.000000,1.000000}%
\pgfsetfillcolor{currentfill}%
\pgfsetlinewidth{0.000000pt}%
\definecolor{currentstroke}{rgb}{0.000000,0.000000,0.000000}%
\pgfsetstrokecolor{currentstroke}%
\pgfsetstrokeopacity{0.000000}%
\pgfsetdash{}{0pt}%
\pgfpathmoveto{\pgfqpoint{4.459969in}{0.261892in}}%
\pgfpathlineto{\pgfqpoint{6.102000in}{0.261892in}}%
\pgfpathlineto{\pgfqpoint{6.102000in}{1.843719in}}%
\pgfpathlineto{\pgfqpoint{4.459969in}{1.843719in}}%
\pgfpathclose%
\pgfusepath{fill}%
\end{pgfscope}%
\begin{pgfscope}%
\pgfpathrectangle{\pgfqpoint{4.459969in}{0.261892in}}{\pgfqpoint{1.642031in}{1.581827in}}%
\pgfusepath{clip}%
\pgfsetrectcap%
\pgfsetroundjoin%
\pgfsetlinewidth{0.281050pt}%
\definecolor{currentstroke}{rgb}{0.690196,0.690196,0.690196}%
\pgfsetstrokecolor{currentstroke}%
\pgfsetdash{}{0pt}%
\pgfpathmoveto{\pgfqpoint{4.556559in}{0.261892in}}%
\pgfpathlineto{\pgfqpoint{4.556559in}{1.843719in}}%
\pgfusepath{stroke}%
\end{pgfscope}%
\begin{pgfscope}%
\pgfsetbuttcap%
\pgfsetroundjoin%
\definecolor{currentfill}{rgb}{0.000000,0.000000,0.000000}%
\pgfsetfillcolor{currentfill}%
\pgfsetlinewidth{0.803000pt}%
\definecolor{currentstroke}{rgb}{0.000000,0.000000,0.000000}%
\pgfsetstrokecolor{currentstroke}%
\pgfsetdash{}{0pt}%
\pgfsys@defobject{currentmarker}{\pgfqpoint{0.000000in}{-0.048611in}}{\pgfqpoint{0.000000in}{0.000000in}}{%
\pgfpathmoveto{\pgfqpoint{0.000000in}{0.000000in}}%
\pgfpathlineto{\pgfqpoint{0.000000in}{-0.048611in}}%
\pgfusepath{stroke,fill}%
}%
\begin{pgfscope}%
\pgfsys@transformshift{4.556559in}{0.261892in}%
\pgfsys@useobject{currentmarker}{}%
\end{pgfscope}%
\end{pgfscope}%
\begin{pgfscope}%
\definecolor{textcolor}{rgb}{0.000000,0.000000,0.000000}%
\pgfsetstrokecolor{textcolor}%
\pgfsetfillcolor{textcolor}%
\pgftext[x=4.556559in,y=0.164670in,,top]{\color{textcolor}\sffamily\fontsize{8.000000}{9.600000}\selectfont −1.0}%
\end{pgfscope}%
\begin{pgfscope}%
\pgfpathrectangle{\pgfqpoint{4.459969in}{0.261892in}}{\pgfqpoint{1.642031in}{1.581827in}}%
\pgfusepath{clip}%
\pgfsetrectcap%
\pgfsetroundjoin%
\pgfsetlinewidth{0.281050pt}%
\definecolor{currentstroke}{rgb}{0.690196,0.690196,0.690196}%
\pgfsetstrokecolor{currentstroke}%
\pgfsetdash{}{0pt}%
\pgfpathmoveto{\pgfqpoint{5.039509in}{0.261892in}}%
\pgfpathlineto{\pgfqpoint{5.039509in}{1.843719in}}%
\pgfusepath{stroke}%
\end{pgfscope}%
\begin{pgfscope}%
\pgfsetbuttcap%
\pgfsetroundjoin%
\definecolor{currentfill}{rgb}{0.000000,0.000000,0.000000}%
\pgfsetfillcolor{currentfill}%
\pgfsetlinewidth{0.803000pt}%
\definecolor{currentstroke}{rgb}{0.000000,0.000000,0.000000}%
\pgfsetstrokecolor{currentstroke}%
\pgfsetdash{}{0pt}%
\pgfsys@defobject{currentmarker}{\pgfqpoint{0.000000in}{-0.048611in}}{\pgfqpoint{0.000000in}{0.000000in}}{%
\pgfpathmoveto{\pgfqpoint{0.000000in}{0.000000in}}%
\pgfpathlineto{\pgfqpoint{0.000000in}{-0.048611in}}%
\pgfusepath{stroke,fill}%
}%
\begin{pgfscope}%
\pgfsys@transformshift{5.039509in}{0.261892in}%
\pgfsys@useobject{currentmarker}{}%
\end{pgfscope}%
\end{pgfscope}%
\begin{pgfscope}%
\definecolor{textcolor}{rgb}{0.000000,0.000000,0.000000}%
\pgfsetstrokecolor{textcolor}%
\pgfsetfillcolor{textcolor}%
\pgftext[x=5.039509in,y=0.164670in,,top]{\color{textcolor}\sffamily\fontsize{8.000000}{9.600000}\selectfont −0.5}%
\end{pgfscope}%
\begin{pgfscope}%
\pgfpathrectangle{\pgfqpoint{4.459969in}{0.261892in}}{\pgfqpoint{1.642031in}{1.581827in}}%
\pgfusepath{clip}%
\pgfsetrectcap%
\pgfsetroundjoin%
\pgfsetlinewidth{0.281050pt}%
\definecolor{currentstroke}{rgb}{0.690196,0.690196,0.690196}%
\pgfsetstrokecolor{currentstroke}%
\pgfsetdash{}{0pt}%
\pgfpathmoveto{\pgfqpoint{5.522460in}{0.261892in}}%
\pgfpathlineto{\pgfqpoint{5.522460in}{1.843719in}}%
\pgfusepath{stroke}%
\end{pgfscope}%
\begin{pgfscope}%
\pgfsetbuttcap%
\pgfsetroundjoin%
\definecolor{currentfill}{rgb}{0.000000,0.000000,0.000000}%
\pgfsetfillcolor{currentfill}%
\pgfsetlinewidth{0.803000pt}%
\definecolor{currentstroke}{rgb}{0.000000,0.000000,0.000000}%
\pgfsetstrokecolor{currentstroke}%
\pgfsetdash{}{0pt}%
\pgfsys@defobject{currentmarker}{\pgfqpoint{0.000000in}{-0.048611in}}{\pgfqpoint{0.000000in}{0.000000in}}{%
\pgfpathmoveto{\pgfqpoint{0.000000in}{0.000000in}}%
\pgfpathlineto{\pgfqpoint{0.000000in}{-0.048611in}}%
\pgfusepath{stroke,fill}%
}%
\begin{pgfscope}%
\pgfsys@transformshift{5.522460in}{0.261892in}%
\pgfsys@useobject{currentmarker}{}%
\end{pgfscope}%
\end{pgfscope}%
\begin{pgfscope}%
\definecolor{textcolor}{rgb}{0.000000,0.000000,0.000000}%
\pgfsetstrokecolor{textcolor}%
\pgfsetfillcolor{textcolor}%
\pgftext[x=5.522460in,y=0.164670in,,top]{\color{textcolor}\sffamily\fontsize{8.000000}{9.600000}\selectfont 0.0}%
\end{pgfscope}%
\begin{pgfscope}%
\pgfpathrectangle{\pgfqpoint{4.459969in}{0.261892in}}{\pgfqpoint{1.642031in}{1.581827in}}%
\pgfusepath{clip}%
\pgfsetrectcap%
\pgfsetroundjoin%
\pgfsetlinewidth{0.281050pt}%
\definecolor{currentstroke}{rgb}{0.690196,0.690196,0.690196}%
\pgfsetstrokecolor{currentstroke}%
\pgfsetdash{}{0pt}%
\pgfpathmoveto{\pgfqpoint{6.005410in}{0.261892in}}%
\pgfpathlineto{\pgfqpoint{6.005410in}{1.843719in}}%
\pgfusepath{stroke}%
\end{pgfscope}%
\begin{pgfscope}%
\pgfsetbuttcap%
\pgfsetroundjoin%
\definecolor{currentfill}{rgb}{0.000000,0.000000,0.000000}%
\pgfsetfillcolor{currentfill}%
\pgfsetlinewidth{0.803000pt}%
\definecolor{currentstroke}{rgb}{0.000000,0.000000,0.000000}%
\pgfsetstrokecolor{currentstroke}%
\pgfsetdash{}{0pt}%
\pgfsys@defobject{currentmarker}{\pgfqpoint{0.000000in}{-0.048611in}}{\pgfqpoint{0.000000in}{0.000000in}}{%
\pgfpathmoveto{\pgfqpoint{0.000000in}{0.000000in}}%
\pgfpathlineto{\pgfqpoint{0.000000in}{-0.048611in}}%
\pgfusepath{stroke,fill}%
}%
\begin{pgfscope}%
\pgfsys@transformshift{6.005410in}{0.261892in}%
\pgfsys@useobject{currentmarker}{}%
\end{pgfscope}%
\end{pgfscope}%
\begin{pgfscope}%
\definecolor{textcolor}{rgb}{0.000000,0.000000,0.000000}%
\pgfsetstrokecolor{textcolor}%
\pgfsetfillcolor{textcolor}%
\pgftext[x=6.005410in,y=0.164670in,,top]{\color{textcolor}\sffamily\fontsize{8.000000}{9.600000}\selectfont 0.5}%
\end{pgfscope}%
\begin{pgfscope}%
\definecolor{textcolor}{rgb}{0.000000,0.000000,0.000000}%
\pgfsetstrokecolor{textcolor}%
\pgfsetfillcolor{textcolor}%
\pgftext[x=5.280984in,y=0.010448in,,top]{\color{textcolor}\sffamily\fontsize{8.000000}{9.600000}\selectfont \(\displaystyle (H-K)_0\)}%
\end{pgfscope}%
\begin{pgfscope}%
\pgfpathrectangle{\pgfqpoint{4.459969in}{0.261892in}}{\pgfqpoint{1.642031in}{1.581827in}}%
\pgfusepath{clip}%
\pgfsetbuttcap%
\pgfsetmiterjoin%
\definecolor{currentfill}{rgb}{0.000000,0.000000,0.000000}%
\pgfsetfillcolor{currentfill}%
\pgfsetlinewidth{1.003750pt}%
\definecolor{currentstroke}{rgb}{0.000000,0.000000,0.000000}%
\pgfsetstrokecolor{currentstroke}%
\pgfsetdash{}{0pt}%
\pgfpathmoveto{\pgfqpoint{6.085933in}{0.795857in}}%
\pgfpathlineto{\pgfqpoint{6.058413in}{0.866554in}}%
\pgfpathlineto{\pgfqpoint{6.040168in}{0.851596in}}%
\pgfpathlineto{\pgfqpoint{5.812602in}{1.125012in}}%
\pgfpathlineto{\pgfqpoint{5.811857in}{1.124401in}}%
\pgfpathlineto{\pgfqpoint{6.039424in}{0.850985in}}%
\pgfpathlineto{\pgfqpoint{6.021179in}{0.836026in}}%
\pgfpathclose%
\pgfusepath{stroke,fill}%
\end{pgfscope}%
\begin{pgfscope}%
\pgfpathrectangle{\pgfqpoint{4.459969in}{0.261892in}}{\pgfqpoint{1.642031in}{1.581827in}}%
\pgfusepath{clip}%
\pgfsetbuttcap%
\pgfsetmiterjoin%
\definecolor{currentfill}{rgb}{0.580392,0.403922,0.741176}%
\pgfsetfillcolor{currentfill}%
\pgfsetlinewidth{1.003750pt}%
\definecolor{currentstroke}{rgb}{0.000000,0.000000,0.000000}%
\pgfsetstrokecolor{currentstroke}%
\pgfsetdash{}{0pt}%
\pgfpathmoveto{\pgfqpoint{5.659708in}{0.887652in}}%
\pgfpathlineto{\pgfqpoint{5.670716in}{0.859373in}}%
\pgfpathlineto{\pgfqpoint{5.677790in}{0.865173in}}%
\pgfpathlineto{\pgfqpoint{5.773221in}{0.750515in}}%
\pgfpathlineto{\pgfqpoint{5.773966in}{0.751126in}}%
\pgfpathlineto{\pgfqpoint{5.678535in}{0.865784in}}%
\pgfpathlineto{\pgfqpoint{5.685609in}{0.871584in}}%
\pgfpathclose%
\pgfusepath{stroke,fill}%
\end{pgfscope}%
\begin{pgfscope}%
\pgfpathrectangle{\pgfqpoint{4.459969in}{0.261892in}}{\pgfqpoint{1.642031in}{1.581827in}}%
\pgfusepath{clip}%
\pgfsetrectcap%
\pgfsetroundjoin%
\pgfsetlinewidth{0.401500pt}%
\definecolor{currentstroke}{rgb}{0.172549,0.627451,0.172549}%
\pgfsetstrokecolor{currentstroke}%
\pgfsetdash{}{0pt}%
\pgfpathmoveto{\pgfqpoint{5.435528in}{1.776611in}}%
\pgfpathlineto{\pgfqpoint{5.435528in}{1.776611in}}%
\pgfpathlineto{\pgfqpoint{5.445188in}{1.767024in}}%
\pgfpathlineto{\pgfqpoint{5.435528in}{1.757438in}}%
\pgfpathlineto{\pgfqpoint{5.445188in}{1.747851in}}%
\pgfpathlineto{\pgfqpoint{5.445188in}{1.738264in}}%
\pgfpathlineto{\pgfqpoint{5.454847in}{1.728677in}}%
\pgfpathlineto{\pgfqpoint{5.474165in}{1.719090in}}%
\pgfpathlineto{\pgfqpoint{5.464506in}{1.690330in}}%
\pgfpathlineto{\pgfqpoint{5.474165in}{1.661569in}}%
\pgfpathlineto{\pgfqpoint{5.483824in}{1.661569in}}%
\pgfpathlineto{\pgfqpoint{5.493483in}{1.642396in}}%
\pgfpathlineto{\pgfqpoint{5.503142in}{1.632809in}}%
\pgfpathlineto{\pgfqpoint{5.512801in}{1.623222in}}%
\pgfpathlineto{\pgfqpoint{5.532119in}{1.594461in}}%
\pgfpathlineto{\pgfqpoint{5.541778in}{1.584875in}}%
\pgfpathlineto{\pgfqpoint{5.541778in}{1.565701in}}%
\pgfpathlineto{\pgfqpoint{5.561096in}{1.556114in}}%
\pgfpathlineto{\pgfqpoint{5.561096in}{1.546527in}}%
\pgfpathlineto{\pgfqpoint{5.580414in}{1.517767in}}%
\pgfpathlineto{\pgfqpoint{5.609391in}{1.460246in}}%
\pgfpathlineto{\pgfqpoint{5.628709in}{1.431485in}}%
\pgfpathlineto{\pgfqpoint{5.638368in}{1.393138in}}%
\pgfpathlineto{\pgfqpoint{5.686663in}{1.316443in}}%
\pgfpathlineto{\pgfqpoint{5.705981in}{1.239749in}}%
\pgfpathlineto{\pgfqpoint{5.705981in}{1.172641in}}%
\pgfpathlineto{\pgfqpoint{5.705981in}{1.153467in}}%
\pgfpathlineto{\pgfqpoint{5.715640in}{1.143880in}}%
\pgfpathlineto{\pgfqpoint{5.705981in}{1.134294in}}%
\pgfpathlineto{\pgfqpoint{5.705981in}{1.105533in}}%
\pgfpathlineto{\pgfqpoint{5.696322in}{1.048012in}}%
\pgfpathlineto{\pgfqpoint{5.677004in}{1.000078in}}%
\pgfpathlineto{\pgfqpoint{5.667345in}{0.980904in}}%
\pgfpathlineto{\pgfqpoint{5.657686in}{0.961731in}}%
\pgfpathlineto{\pgfqpoint{5.628709in}{0.904210in}}%
\pgfpathlineto{\pgfqpoint{5.609391in}{0.885036in}}%
\pgfpathlineto{\pgfqpoint{5.590073in}{0.856275in}}%
\pgfpathlineto{\pgfqpoint{5.561096in}{0.827515in}}%
\pgfpathlineto{\pgfqpoint{5.474165in}{0.741233in}}%
\pgfpathlineto{\pgfqpoint{5.454847in}{0.722060in}}%
\pgfpathlineto{\pgfqpoint{5.425869in}{0.693299in}}%
\pgfpathlineto{\pgfqpoint{5.396892in}{0.674126in}}%
\pgfpathlineto{\pgfqpoint{5.358256in}{0.645365in}}%
\pgfpathlineto{\pgfqpoint{5.271325in}{0.587844in}}%
\pgfpathlineto{\pgfqpoint{5.165076in}{0.520736in}}%
\pgfpathlineto{\pgfqpoint{5.039509in}{0.444042in}}%
\pgfpathlineto{\pgfqpoint{4.904283in}{0.376934in}}%
\pgfpathlineto{\pgfqpoint{4.617883in}{0.248559in}}%
\pgfusepath{stroke}%
\end{pgfscope}%
\begin{pgfscope}%
\pgfpathrectangle{\pgfqpoint{4.459969in}{0.261892in}}{\pgfqpoint{1.642031in}{1.581827in}}%
\pgfusepath{clip}%
\pgfsetbuttcap%
\pgfsetroundjoin%
\definecolor{currentfill}{rgb}{0.172549,0.627451,0.172549}%
\pgfsetfillcolor{currentfill}%
\pgfsetlinewidth{1.003750pt}%
\definecolor{currentstroke}{rgb}{0.172549,0.627451,0.172549}%
\pgfsetstrokecolor{currentstroke}%
\pgfsetdash{}{0pt}%
\pgfsys@defobject{currentmarker}{\pgfqpoint{-0.006944in}{-0.006944in}}{\pgfqpoint{0.006944in}{0.006944in}}{%
\pgfpathmoveto{\pgfqpoint{0.000000in}{-0.006944in}}%
\pgfpathcurveto{\pgfqpoint{0.001842in}{-0.006944in}}{\pgfqpoint{0.003608in}{-0.006213in}}{\pgfqpoint{0.004910in}{-0.004910in}}%
\pgfpathcurveto{\pgfqpoint{0.006213in}{-0.003608in}}{\pgfqpoint{0.006944in}{-0.001842in}}{\pgfqpoint{0.006944in}{0.000000in}}%
\pgfpathcurveto{\pgfqpoint{0.006944in}{0.001842in}}{\pgfqpoint{0.006213in}{0.003608in}}{\pgfqpoint{0.004910in}{0.004910in}}%
\pgfpathcurveto{\pgfqpoint{0.003608in}{0.006213in}}{\pgfqpoint{0.001842in}{0.006944in}}{\pgfqpoint{0.000000in}{0.006944in}}%
\pgfpathcurveto{\pgfqpoint{-0.001842in}{0.006944in}}{\pgfqpoint{-0.003608in}{0.006213in}}{\pgfqpoint{-0.004910in}{0.004910in}}%
\pgfpathcurveto{\pgfqpoint{-0.006213in}{0.003608in}}{\pgfqpoint{-0.006944in}{0.001842in}}{\pgfqpoint{-0.006944in}{0.000000in}}%
\pgfpathcurveto{\pgfqpoint{-0.006944in}{-0.001842in}}{\pgfqpoint{-0.006213in}{-0.003608in}}{\pgfqpoint{-0.004910in}{-0.004910in}}%
\pgfpathcurveto{\pgfqpoint{-0.003608in}{-0.006213in}}{\pgfqpoint{-0.001842in}{-0.006944in}}{\pgfqpoint{0.000000in}{-0.006944in}}%
\pgfpathclose%
\pgfusepath{stroke,fill}%
}%
\begin{pgfscope}%
\pgfsys@transformshift{5.435528in}{1.776611in}%
\pgfsys@useobject{currentmarker}{}%
\end{pgfscope}%
\begin{pgfscope}%
\pgfsys@transformshift{5.435528in}{1.776611in}%
\pgfsys@useobject{currentmarker}{}%
\end{pgfscope}%
\begin{pgfscope}%
\pgfsys@transformshift{5.445188in}{1.767024in}%
\pgfsys@useobject{currentmarker}{}%
\end{pgfscope}%
\begin{pgfscope}%
\pgfsys@transformshift{5.435528in}{1.757438in}%
\pgfsys@useobject{currentmarker}{}%
\end{pgfscope}%
\begin{pgfscope}%
\pgfsys@transformshift{5.445188in}{1.747851in}%
\pgfsys@useobject{currentmarker}{}%
\end{pgfscope}%
\begin{pgfscope}%
\pgfsys@transformshift{5.445188in}{1.738264in}%
\pgfsys@useobject{currentmarker}{}%
\end{pgfscope}%
\begin{pgfscope}%
\pgfsys@transformshift{5.454847in}{1.728677in}%
\pgfsys@useobject{currentmarker}{}%
\end{pgfscope}%
\begin{pgfscope}%
\pgfsys@transformshift{5.474165in}{1.719090in}%
\pgfsys@useobject{currentmarker}{}%
\end{pgfscope}%
\begin{pgfscope}%
\pgfsys@transformshift{5.464506in}{1.690330in}%
\pgfsys@useobject{currentmarker}{}%
\end{pgfscope}%
\begin{pgfscope}%
\pgfsys@transformshift{5.474165in}{1.661569in}%
\pgfsys@useobject{currentmarker}{}%
\end{pgfscope}%
\begin{pgfscope}%
\pgfsys@transformshift{5.483824in}{1.661569in}%
\pgfsys@useobject{currentmarker}{}%
\end{pgfscope}%
\begin{pgfscope}%
\pgfsys@transformshift{5.493483in}{1.642396in}%
\pgfsys@useobject{currentmarker}{}%
\end{pgfscope}%
\begin{pgfscope}%
\pgfsys@transformshift{5.503142in}{1.632809in}%
\pgfsys@useobject{currentmarker}{}%
\end{pgfscope}%
\begin{pgfscope}%
\pgfsys@transformshift{5.512801in}{1.623222in}%
\pgfsys@useobject{currentmarker}{}%
\end{pgfscope}%
\begin{pgfscope}%
\pgfsys@transformshift{5.532119in}{1.594461in}%
\pgfsys@useobject{currentmarker}{}%
\end{pgfscope}%
\begin{pgfscope}%
\pgfsys@transformshift{5.541778in}{1.584875in}%
\pgfsys@useobject{currentmarker}{}%
\end{pgfscope}%
\begin{pgfscope}%
\pgfsys@transformshift{5.541778in}{1.565701in}%
\pgfsys@useobject{currentmarker}{}%
\end{pgfscope}%
\begin{pgfscope}%
\pgfsys@transformshift{5.561096in}{1.556114in}%
\pgfsys@useobject{currentmarker}{}%
\end{pgfscope}%
\begin{pgfscope}%
\pgfsys@transformshift{5.561096in}{1.546527in}%
\pgfsys@useobject{currentmarker}{}%
\end{pgfscope}%
\begin{pgfscope}%
\pgfsys@transformshift{5.580414in}{1.517767in}%
\pgfsys@useobject{currentmarker}{}%
\end{pgfscope}%
\begin{pgfscope}%
\pgfsys@transformshift{5.609391in}{1.460246in}%
\pgfsys@useobject{currentmarker}{}%
\end{pgfscope}%
\begin{pgfscope}%
\pgfsys@transformshift{5.628709in}{1.431485in}%
\pgfsys@useobject{currentmarker}{}%
\end{pgfscope}%
\begin{pgfscope}%
\pgfsys@transformshift{5.638368in}{1.393138in}%
\pgfsys@useobject{currentmarker}{}%
\end{pgfscope}%
\begin{pgfscope}%
\pgfsys@transformshift{5.686663in}{1.316443in}%
\pgfsys@useobject{currentmarker}{}%
\end{pgfscope}%
\begin{pgfscope}%
\pgfsys@transformshift{5.705981in}{1.239749in}%
\pgfsys@useobject{currentmarker}{}%
\end{pgfscope}%
\begin{pgfscope}%
\pgfsys@transformshift{5.705981in}{1.172641in}%
\pgfsys@useobject{currentmarker}{}%
\end{pgfscope}%
\begin{pgfscope}%
\pgfsys@transformshift{5.705981in}{1.153467in}%
\pgfsys@useobject{currentmarker}{}%
\end{pgfscope}%
\begin{pgfscope}%
\pgfsys@transformshift{5.715640in}{1.143880in}%
\pgfsys@useobject{currentmarker}{}%
\end{pgfscope}%
\begin{pgfscope}%
\pgfsys@transformshift{5.705981in}{1.134294in}%
\pgfsys@useobject{currentmarker}{}%
\end{pgfscope}%
\begin{pgfscope}%
\pgfsys@transformshift{5.705981in}{1.105533in}%
\pgfsys@useobject{currentmarker}{}%
\end{pgfscope}%
\begin{pgfscope}%
\pgfsys@transformshift{5.696322in}{1.048012in}%
\pgfsys@useobject{currentmarker}{}%
\end{pgfscope}%
\begin{pgfscope}%
\pgfsys@transformshift{5.677004in}{1.000078in}%
\pgfsys@useobject{currentmarker}{}%
\end{pgfscope}%
\begin{pgfscope}%
\pgfsys@transformshift{5.667345in}{0.980904in}%
\pgfsys@useobject{currentmarker}{}%
\end{pgfscope}%
\begin{pgfscope}%
\pgfsys@transformshift{5.657686in}{0.961731in}%
\pgfsys@useobject{currentmarker}{}%
\end{pgfscope}%
\begin{pgfscope}%
\pgfsys@transformshift{5.628709in}{0.904210in}%
\pgfsys@useobject{currentmarker}{}%
\end{pgfscope}%
\begin{pgfscope}%
\pgfsys@transformshift{5.609391in}{0.885036in}%
\pgfsys@useobject{currentmarker}{}%
\end{pgfscope}%
\begin{pgfscope}%
\pgfsys@transformshift{5.590073in}{0.856275in}%
\pgfsys@useobject{currentmarker}{}%
\end{pgfscope}%
\begin{pgfscope}%
\pgfsys@transformshift{5.561096in}{0.827515in}%
\pgfsys@useobject{currentmarker}{}%
\end{pgfscope}%
\begin{pgfscope}%
\pgfsys@transformshift{5.474165in}{0.741233in}%
\pgfsys@useobject{currentmarker}{}%
\end{pgfscope}%
\begin{pgfscope}%
\pgfsys@transformshift{5.454847in}{0.722060in}%
\pgfsys@useobject{currentmarker}{}%
\end{pgfscope}%
\begin{pgfscope}%
\pgfsys@transformshift{5.425869in}{0.693299in}%
\pgfsys@useobject{currentmarker}{}%
\end{pgfscope}%
\begin{pgfscope}%
\pgfsys@transformshift{5.396892in}{0.674126in}%
\pgfsys@useobject{currentmarker}{}%
\end{pgfscope}%
\begin{pgfscope}%
\pgfsys@transformshift{5.358256in}{0.645365in}%
\pgfsys@useobject{currentmarker}{}%
\end{pgfscope}%
\begin{pgfscope}%
\pgfsys@transformshift{5.271325in}{0.587844in}%
\pgfsys@useobject{currentmarker}{}%
\end{pgfscope}%
\begin{pgfscope}%
\pgfsys@transformshift{5.165076in}{0.520736in}%
\pgfsys@useobject{currentmarker}{}%
\end{pgfscope}%
\begin{pgfscope}%
\pgfsys@transformshift{5.039509in}{0.444042in}%
\pgfsys@useobject{currentmarker}{}%
\end{pgfscope}%
\begin{pgfscope}%
\pgfsys@transformshift{4.904283in}{0.376934in}%
\pgfsys@useobject{currentmarker}{}%
\end{pgfscope}%
\begin{pgfscope}%
\pgfsys@transformshift{4.604854in}{0.242718in}%
\pgfsys@useobject{currentmarker}{}%
\end{pgfscope}%
\end{pgfscope}%
\begin{pgfscope}%
\pgfpathrectangle{\pgfqpoint{4.459969in}{0.261892in}}{\pgfqpoint{1.642031in}{1.581827in}}%
\pgfusepath{clip}%
\pgfsetbuttcap%
\pgfsetbeveljoin%
\definecolor{currentfill}{rgb}{0.580392,0.403922,0.741176}%
\pgfsetfillcolor{currentfill}%
\pgfsetlinewidth{1.003750pt}%
\definecolor{currentstroke}{rgb}{0.580392,0.403922,0.741176}%
\pgfsetstrokecolor{currentstroke}%
\pgfsetdash{}{0pt}%
\pgfsys@defobject{currentmarker}{\pgfqpoint{-0.033023in}{-0.028091in}}{\pgfqpoint{0.033023in}{0.034722in}}{%
\pgfpathmoveto{\pgfqpoint{0.000000in}{0.034722in}}%
\pgfpathlineto{\pgfqpoint{-0.007796in}{0.010730in}}%
\pgfpathlineto{\pgfqpoint{-0.033023in}{0.010730in}}%
\pgfpathlineto{\pgfqpoint{-0.012614in}{-0.004098in}}%
\pgfpathlineto{\pgfqpoint{-0.020409in}{-0.028091in}}%
\pgfpathlineto{\pgfqpoint{-0.000000in}{-0.013263in}}%
\pgfpathlineto{\pgfqpoint{0.020409in}{-0.028091in}}%
\pgfpathlineto{\pgfqpoint{0.012614in}{-0.004098in}}%
\pgfpathlineto{\pgfqpoint{0.033023in}{0.010730in}}%
\pgfpathlineto{\pgfqpoint{0.007796in}{0.010730in}}%
\pgfpathclose%
\pgfusepath{stroke,fill}%
}%
\begin{pgfscope}%
\pgfsys@transformshift{5.773594in}{0.750820in}%
\pgfsys@useobject{currentmarker}{}%
\end{pgfscope}%
\end{pgfscope}%
\begin{pgfscope}%
\pgfsetrectcap%
\pgfsetmiterjoin%
\pgfsetlinewidth{0.803000pt}%
\definecolor{currentstroke}{rgb}{0.000000,0.000000,0.000000}%
\pgfsetstrokecolor{currentstroke}%
\pgfsetdash{}{0pt}%
\pgfpathmoveto{\pgfqpoint{4.459969in}{0.261892in}}%
\pgfpathlineto{\pgfqpoint{4.459969in}{1.843719in}}%
\pgfusepath{stroke}%
\end{pgfscope}%
\begin{pgfscope}%
\pgfsetrectcap%
\pgfsetmiterjoin%
\pgfsetlinewidth{0.803000pt}%
\definecolor{currentstroke}{rgb}{0.000000,0.000000,0.000000}%
\pgfsetstrokecolor{currentstroke}%
\pgfsetdash{}{0pt}%
\pgfpathmoveto{\pgfqpoint{6.102000in}{0.261892in}}%
\pgfpathlineto{\pgfqpoint{6.102000in}{1.843719in}}%
\pgfusepath{stroke}%
\end{pgfscope}%
\begin{pgfscope}%
\pgfsetrectcap%
\pgfsetmiterjoin%
\pgfsetlinewidth{0.803000pt}%
\definecolor{currentstroke}{rgb}{0.000000,0.000000,0.000000}%
\pgfsetstrokecolor{currentstroke}%
\pgfsetdash{}{0pt}%
\pgfpathmoveto{\pgfqpoint{4.459969in}{0.261892in}}%
\pgfpathlineto{\pgfqpoint{6.102000in}{0.261892in}}%
\pgfusepath{stroke}%
\end{pgfscope}%
\begin{pgfscope}%
\pgfsetrectcap%
\pgfsetmiterjoin%
\pgfsetlinewidth{0.803000pt}%
\definecolor{currentstroke}{rgb}{0.000000,0.000000,0.000000}%
\pgfsetstrokecolor{currentstroke}%
\pgfsetdash{}{0pt}%
\pgfpathmoveto{\pgfqpoint{4.459969in}{1.843719in}}%
\pgfpathlineto{\pgfqpoint{6.102000in}{1.843719in}}%
\pgfusepath{stroke}%
\end{pgfscope}%
\begin{pgfscope}%
\pgfsetbuttcap%
\pgfsetmiterjoin%
\definecolor{currentfill}{rgb}{1.000000,1.000000,1.000000}%
\pgfsetfillcolor{currentfill}%
\pgfsetfillopacity{0.800000}%
\pgfsetlinewidth{1.003750pt}%
\definecolor{currentstroke}{rgb}{0.800000,0.800000,0.800000}%
\pgfsetstrokecolor{currentstroke}%
\pgfsetstrokeopacity{0.800000}%
\pgfsetdash{}{0pt}%
\pgfpathmoveto{\pgfqpoint{4.518302in}{0.922472in}}%
\pgfpathlineto{\pgfqpoint{5.231969in}{0.922472in}}%
\pgfpathquadraticcurveto{\pgfqpoint{5.248635in}{0.922472in}}{\pgfqpoint{5.248635in}{0.939139in}}%
\pgfpathlineto{\pgfqpoint{5.248635in}{1.166472in}}%
\pgfpathquadraticcurveto{\pgfqpoint{5.248635in}{1.183139in}}{\pgfqpoint{5.231969in}{1.183139in}}%
\pgfpathlineto{\pgfqpoint{4.518302in}{1.183139in}}%
\pgfpathquadraticcurveto{\pgfqpoint{4.501635in}{1.183139in}}{\pgfqpoint{4.501635in}{1.166472in}}%
\pgfpathlineto{\pgfqpoint{4.501635in}{0.939139in}}%
\pgfpathquadraticcurveto{\pgfqpoint{4.501635in}{0.922472in}}{\pgfqpoint{4.518302in}{0.922472in}}%
\pgfpathclose%
\pgfusepath{stroke,fill}%
\end{pgfscope}%
\begin{pgfscope}%
\pgfsetrectcap%
\pgfsetroundjoin%
\pgfsetlinewidth{0.401500pt}%
\definecolor{currentstroke}{rgb}{0.172549,0.627451,0.172549}%
\pgfsetstrokecolor{currentstroke}%
\pgfsetdash{}{0pt}%
\pgfpathmoveto{\pgfqpoint{4.534969in}{1.119389in}}%
\pgfpathlineto{\pgfqpoint{4.701635in}{1.119389in}}%
\pgfusepath{stroke}%
\end{pgfscope}%
\begin{pgfscope}%
\pgfsetbuttcap%
\pgfsetroundjoin%
\definecolor{currentfill}{rgb}{0.172549,0.627451,0.172549}%
\pgfsetfillcolor{currentfill}%
\pgfsetlinewidth{1.003750pt}%
\definecolor{currentstroke}{rgb}{0.172549,0.627451,0.172549}%
\pgfsetstrokecolor{currentstroke}%
\pgfsetdash{}{0pt}%
\pgfsys@defobject{currentmarker}{\pgfqpoint{-0.006944in}{-0.006944in}}{\pgfqpoint{0.006944in}{0.006944in}}{%
\pgfpathmoveto{\pgfqpoint{0.000000in}{-0.006944in}}%
\pgfpathcurveto{\pgfqpoint{0.001842in}{-0.006944in}}{\pgfqpoint{0.003608in}{-0.006213in}}{\pgfqpoint{0.004910in}{-0.004910in}}%
\pgfpathcurveto{\pgfqpoint{0.006213in}{-0.003608in}}{\pgfqpoint{0.006944in}{-0.001842in}}{\pgfqpoint{0.006944in}{0.000000in}}%
\pgfpathcurveto{\pgfqpoint{0.006944in}{0.001842in}}{\pgfqpoint{0.006213in}{0.003608in}}{\pgfqpoint{0.004910in}{0.004910in}}%
\pgfpathcurveto{\pgfqpoint{0.003608in}{0.006213in}}{\pgfqpoint{0.001842in}{0.006944in}}{\pgfqpoint{0.000000in}{0.006944in}}%
\pgfpathcurveto{\pgfqpoint{-0.001842in}{0.006944in}}{\pgfqpoint{-0.003608in}{0.006213in}}{\pgfqpoint{-0.004910in}{0.004910in}}%
\pgfpathcurveto{\pgfqpoint{-0.006213in}{0.003608in}}{\pgfqpoint{-0.006944in}{0.001842in}}{\pgfqpoint{-0.006944in}{0.000000in}}%
\pgfpathcurveto{\pgfqpoint{-0.006944in}{-0.001842in}}{\pgfqpoint{-0.006213in}{-0.003608in}}{\pgfqpoint{-0.004910in}{-0.004910in}}%
\pgfpathcurveto{\pgfqpoint{-0.003608in}{-0.006213in}}{\pgfqpoint{-0.001842in}{-0.006944in}}{\pgfqpoint{0.000000in}{-0.006944in}}%
\pgfpathclose%
\pgfusepath{stroke,fill}%
}%
\begin{pgfscope}%
\pgfsys@transformshift{4.618302in}{1.119389in}%
\pgfsys@useobject{currentmarker}{}%
\end{pgfscope}%
\end{pgfscope}%
\begin{pgfscope}%
\definecolor{textcolor}{rgb}{0.000000,0.000000,0.000000}%
\pgfsetstrokecolor{textcolor}%
\pgfsetfillcolor{textcolor}%
\pgftext[x=4.768302in,y=1.090222in,left,base]{\color{textcolor}\sffamily\fontsize{6.000000}{7.200000}\selectfont SuperGiants}%
\end{pgfscope}%
\begin{pgfscope}%
\pgfsetbuttcap%
\pgfsetbeveljoin%
\definecolor{currentfill}{rgb}{0.580392,0.403922,0.741176}%
\pgfsetfillcolor{currentfill}%
\pgfsetlinewidth{1.003750pt}%
\definecolor{currentstroke}{rgb}{0.580392,0.403922,0.741176}%
\pgfsetstrokecolor{currentstroke}%
\pgfsetdash{}{0pt}%
\pgfsys@defobject{currentmarker}{\pgfqpoint{-0.033023in}{-0.028091in}}{\pgfqpoint{0.033023in}{0.034722in}}{%
\pgfpathmoveto{\pgfqpoint{0.000000in}{0.034722in}}%
\pgfpathlineto{\pgfqpoint{-0.007796in}{0.010730in}}%
\pgfpathlineto{\pgfqpoint{-0.033023in}{0.010730in}}%
\pgfpathlineto{\pgfqpoint{-0.012614in}{-0.004098in}}%
\pgfpathlineto{\pgfqpoint{-0.020409in}{-0.028091in}}%
\pgfpathlineto{\pgfqpoint{-0.000000in}{-0.013263in}}%
\pgfpathlineto{\pgfqpoint{0.020409in}{-0.028091in}}%
\pgfpathlineto{\pgfqpoint{0.012614in}{-0.004098in}}%
\pgfpathlineto{\pgfqpoint{0.033023in}{0.010730in}}%
\pgfpathlineto{\pgfqpoint{0.007796in}{0.010730in}}%
\pgfpathclose%
\pgfusepath{stroke,fill}%
}%
\begin{pgfscope}%
\pgfsys@transformshift{4.618302in}{1.001972in}%
\pgfsys@useobject{currentmarker}{}%
\end{pgfscope}%
\end{pgfscope}%
\begin{pgfscope}%
\definecolor{textcolor}{rgb}{0.000000,0.000000,0.000000}%
\pgfsetstrokecolor{textcolor}%
\pgfsetfillcolor{textcolor}%
\pgftext[x=4.768302in,y=0.972805in,left,base]{\color{textcolor}\sffamily\fontsize{6.000000}{7.200000}\selectfont NIR Object}%
\end{pgfscope}%
\end{pgfpicture}%
\makeatother%
\endgroup%

    \caption{Color-Color diagram}
    \label{fig:ColorColorDiagram}
\end{figure}

\subsection{}
\textbf{Assuming that the near-IR object is either a main sequence star or a giant star, estimate or constrain its extinction and compare it with the value inferred from the X-ray absorbing gas column density toward the X-ray source (using the $N_H/A_V = 1.8\times 10^{21} \text{atoms cm}^{-2} \text{ mag}^{-1}$).
Is the near-IR object likely to be the counterpart of the X-ray source?
Similar methods may be used to probe the nature of other types of objects (e.g., galaxies, accounting for extinction and red-shift effects).}

From the column density, we can determined that the extinction of the X-ray is given by 
\begin{equation*}
    A_v = \frac{N_H}{(N_H/A_V)} = \frac{(5\pm0.5)\times10^{22} \si{\per\cm\squared}}{1.8\times 10^{21} \text{ cm}^{-2} \text{ mag}^{-1}} = \num{27.777}\pm2.7 
\end{equation*}



%==============================================================
\newpage
\section{}(10 points)
\textbf{Assuming that the star formation rate for the Galaxy is $6 M_\odot \si{\per\year}$ for the last 1 Gyr or so.}
\subsection{} 
\textbf{Please express and plot an appropriately-scaled differential Salpeter initial mass function (IMF) with an index of $x = 2.35$ for $M \geq 0.5M_\odot$ and $x = 1.3$ for
$0.1M_\odot\leq M < 0.5M_\odot$; the IMF should have a continuity at the break point.}

The Salpeter initial mass function will be given by
\begin{equation*}
    IMF = \frac{dn}{dM} = 
    \begin{cases} 
      A(M/M_\odot)^{-1.3}     & 0.1M_\odot < M < 0.1M_\odot\\
      B(M/M_\odot)^{-2.35}    & M \geq 0.5M_\odot
   \end{cases}
\end{equation*}
where A and B are scaling coefficients. 

Given that both equations need to be continuous at $0.5M_\odot$ we can find a relationship between the coefficients by
\begin{equation*}
    A(0.5)^{-1.3}=B(0.5)^{-2.35} \qquad\rightarrow\qquad A = \frac{B(0.5)^{-2.35}}{(0.5)^{-1.3}} = B*0.5^{-1.05}
\end{equation*}

Integrating the IMF from $0.1M_\odot$ to $100M_\odot$ and substituting A, we get
\begin{align*}
    n &= \int_{0.1M_\odot}^{0.5M_\odot}A(M/M_\odot)^{-1.3}dM + \int_{0.5M_\odot}^{100 M_\odot}B(M/M_\odot)^{-2.35}dM 
    =\left[\frac{A(M/M_\odot)^{-0.3}}{-0.3}\right]_{0.1M_\odot}^{0.5M_\odot} + \left[\frac{B(M/M_\odot)^{-1.35}}{-1.35}\right]_{0.5M_\odot}^{100 M_\odot}\\
    &= \frac{A}{-0.3}(0.5^{-0.3}-0.1^{-0.3})+\frac{B}{-1.35}(100^{-1.35}-0.5^{-1.35})
    = 2.5470A + 1.8867B \\
    &= 2.5407(B*0.5^{-1.05}) + 1.8867B = 7.1605B
\end{align*}
So given that n is equal to the star formation rate, we can find that
\begin{align*}
    B = 6 /1.8867 = 0.8379 ,\text{ and}\\
    A = B*0.5^{-1.05} = 1.7349 
\end{align*}

So the final IMF is given by
\begin{equation*}
    IMF = \frac{dn}{dM} = 
    \begin{cases} 
      1.7349(M/M_\odot)^{-1.3}     & 0.1M_\odot < M < 0.1M_\odot\\
      0.8379(M/M_\odot)^{-2.35}    & M \geq 0.5M_\odot
   \end{cases}
\end{equation*}

Figure \ref{fig:IMF} shows 

\begin{figure}
    \centering
    \input{CodeAndFigures/Astro643_HW4P5plot.pdf}
    \caption{Caption}
    \label{fig:IMF}
\end{figure}


\subsection{}
Further assuming that all stars between $8$ and $100M_\odot$ explode as core-collapsed supernovae, estimate the corresponding present supernova rate. 
How much difference does it make if the upper mass limit for supernovae is assumed to be infinite? (The observed rate for
core-collapsed supernovae in galaxies like our own is of the order of 2 to 3 per century.)




